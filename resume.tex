\documentclass[10pt]{article}

\usepackage{marvosym}
\usepackage{fontspec}
\usepackage{xunicode,xltxtra,url,parskip}
\defaultfontfeatures{Scale=MatchLowercase,Mapping=tex-text}
\RequirePackage{color,graphics}
\usepackage[usenames,dvipsnames]{xcolor}
\usepackage[left=1in, right=1in, top=0.5in, bottom=0.5in]{geometry}
\usepackage{supertabular}
\usepackage{titlesec}
\usepackage{multicol}
\usepackage{multirow}
\usepackage{longtable}
\usepackage{xstring}
\usepackage{ifthen}
\usepackage{enumitem}
\usepackage[xetex,
            unicode,
            pdfencoding=auto,
            pdfinfo={
              Title={malloc47/resume},
              Author={Jarrell Waggoner},
              Subject={Jarrell Waggoner Résumé},
              Keywords={computer vision, image processing, artificial intelligence, pattern recognition, machine learning, data science, functional programming, web development},
              Producer={xelatex},
              Creator{xelatex}
            },
            ]{hyperref}
\usepackage[absolute]{textpos}
\usepackage{enumitem}
\usepackage{tabularx}

% \makeatletter
% \renewcommand*{\@biblabel}[1]{\hfill[#1]}
% \makeatother

% Template obtained from http://www.cv-templates.info/2009/03/professional-cv-latex/

%Setup hyperref package, and colours for links
\definecolor{linkcolour}{rgb}{0,0.2,0.6}
\hypersetup{colorlinks,breaklinks,urlcolor=linkcolour, linkcolor=linkcolour}

%Color
\definecolor{lightg}{HTML}{999999}
\definecolor{medg}{HTML}{666666}
\definecolor{darkg}{HTML}{333333}

% Bullets
\definecolor{noteone}{HTML}{999999}
\definecolor{notetwo}{HTML}{848484}
\definecolor{notethree}{HTML}{424242}
\definecolor{notefour}{HTML}{212121}
\definecolor{notefive}{HTML}{000000}

\newcommand{\fivenotes}{%
	\textcolor{noteone}{\symbol{"2022}}
	\textcolor{notetwo}{\symbol{"2022}}
	\textcolor{notethree}{\symbol{"2022}}
	\textcolor{notefour}{\symbol{"2022}}
	\textcolor{notefive}{\symbol{"2022}}
}
\newcommand{\fournotes}{%
	\textcolor{noteone}{\symbol{"2022}}
	\textcolor{notetwo}{\symbol{"2022}}
	\textcolor{notethree}{\symbol{"2022}}
	\textcolor{notefour}{\symbol{"2022}}
	\textcolor{white}{\symbol{"2022}}
}
\newcommand{\threenotes}{%
	\textcolor{noteone}{\symbol{"2022}}
	\textcolor{notetwo}{\symbol{"2022}}
	\textcolor{notethree}{\symbol{"2022}}
	\textcolor{white}{\symbol{"2022}}
	\textcolor{white}{\symbol{"2022}}
}
\newcommand{\twonotes}{%
	\textcolor{noteone}{\symbol{"2022}}
	\textcolor{notetwo}{\symbol{"2022}}
	\textcolor{white}{\symbol{"2022}}
	\textcolor{white}{\symbol{"2022}}
	\textcolor{white}{\symbol{"2022}}
}
\newcommand{\onenote}{%
	\textcolor{noteone}{\symbol{"2022}}
	\textcolor{white}{\symbol{"2022}}
	\textcolor{white}{\symbol{"2022}}
	\textcolor{white}{\symbol{"2022}}
	\textcolor{white}{\symbol{"2022}}
}

\newcommand{\oneskill}{%
  \textcolor{white}{\symbol{"2022}}
  \textcolor{white}{\symbol{"2022}}
  \textcolor{notefive}{\symbol{"2022}}
}

\newcommand{\twoskill}{%
  \textcolor{white}{\symbol{"2022}}
  \textcolor{notethree}{\symbol{"2022}}
  \textcolor{notefive}{\symbol{"2022}}
}

\newcommand{\threeskill}{%
  \textcolor{noteone}{\symbol{"2022}}
  \textcolor{notethree}{\symbol{"2022}}
  \textcolor{notefive}{\symbol{"2022}}
}

%FONTS
% \defaultfontfeatures{Mapping=tex-text}
% \setmainfont[SmallCapsFont = Fontin SmallCaps]{Fontin}

\setromanfont [Ligatures={Common}, BoldFont={Linux Libertine Bold}, ItalicFont={Linux Libertine Italic}]{Linux Libertine}
\setsansfont [Ligatures={Common}, BoldFont={GeosansLight}, ItalicFont={GeosansLight}]{GeosansLight}
\setmonofont{GeosansLight} 

\font\lighttext=''Baskerville-Normal:color=787878'' at 10pt
\font\lighttextweb=''Baskerville-Normal:color=FF1493'' at 10pt

%CV Sections inspired by: 
%http://stefano.italians.nl/archives/26
\titleformat{\section}{\Large\scshape\raggedright\sffamily}{}{0em}{}[\titlerule]
%% \titlespacing{\section}{0pt}{-2pt}{0pt}

%-------------WATERMARK TEST [**not part of a CV**]---------------
\TPGrid[30mm,30mm]{30}{60}
%\setlength{\TPHorizModule}{30mm}
%\setlength{\TPVertModule}{\TPHorizModule}
%\textblockorigin{2mm}{0.65\paperheight}
\setlength{\parindent}{0pt}

\newcommand{\skill}{\textbf}
\newcommand{\institution}{\textsc}

% \def\bullet{\textcolor{medg}{\symbol{"00BB}}}
\def\div{\,\textbar{}\,}

\titlespacing{\section}{0pt}{-2pt}{0pt}

\begin{document}
\pagestyle{empty}
% \font\fb=''[cmr10]''

\par{\centering {\Huge Jarrell \textsc{Waggoner} }\bigskip\par}

\begin{multicols}{2}
\setlength{\parskip}{0pt}
\section{Biographical}
\begin{tabularx}{\linewidth}{@{}l X@{}}
  \textsc{Address}	& \footnotesize{Department of Computer Science and Engineering,} \\
  & \footnotesize{University of South Carolina, Columbia, SC 29208} \\
  \textsc{Phone}       & 847-261-4747\\
  \textsc{email}       & \href{mailto:jarrell.waggoner@gmail.com}{jarrell.waggoner@gmail.com} \\
\end{tabularx}

\vfill
\columnbreak

\section{Online}
\begin{tabularx}{\linewidth}{@{}l X@{}}
  \textsc{Website}	& \href{http://www.malloc47.com}{www.malloc47.com} \\
  \textsc{Twitter}     & \href{https://twitter.com/malloc47}{@malloc47} \\
  \textsc{github}      & \href{http://www.github.com/malloc47}{github.com/malloc47}\\
  \textsc{LinkedIn}    & \href{http://www.linkedin.com/in/malloc47}{linkedin.com/in/malloc47} \\
\end{tabularx}

\end{multicols}

\begin{tabularx}{\textwidth}{@{}l X}
  \textsc{Interests} & computer vision, image processing, artificial
  intelligence, pattern recognition \& machine learning, data science,
  functional programming, web development
\end{tabularx}

\newcommand{\degree}[4]{\textsc{#1} & \textbf{#2} & \textsc{#3} & \textbf{#4}\\}

\section{Education}
\begin{tabular*}{\textwidth}{@{\extracolsep{\fill}}r l p{5.5cm} r}

  \degree{Expected Aug. 2013}%
  {Ph.D.}%
  {Computer Science \& Engineering}%
  {University of South Carolina}

  \degree{May 2009}%
  {M.E.}%
  {Computer Science \& Engineering}%
  {University of South Carolina}

  % \degree{May 2006}%
  % {B.S.}%
  % {Computer Science}%
  % {Bryan College}

  % \degree{May 2004}%
  % {A.S.}%
  % {Computer Science}%
  % {University of South Carolina at Lancaster}

\end{tabular*}

\newcommand{\experience}[5]{
\textsc{#1} & \textbf{#2} #3 \textsc{#4}\\
\nopagebreak &\multicolumn{2}{p{5.5in}}{\small{#5}}\\
\nopagebreak \multicolumn{3}{c}{} \\ [-1ex]
}

\newcommand{\experiencel}[5]{
\textsc{#1} & \textbf{#2} #3 \textsc{#4}\\
\nopagebreak &\multicolumn{2}{p{5.5in}}{\small{#5}}\\
}

\section{Experience}

\setlength\LTleft{0pt}
\setlength\LTright{0pt}
\vspace{-0.5em}
\begin{longtable}{@{\extracolsep{\fill}} l | l r}

  \experience{2012---Present}%
  {Technical Lead}%
  {at}%
  {\href{http://www.huntstand.com}{Huntstand, Inc.}}%
  {Software developer in an agile startup environment creating the
    \href{http://www.huntstand.com}{huntstand.com} web application.
    Written using \python, \django, and \backbone; deployed to
    \skill{AWS}.  Responsible for curating full technology stack and
    coordinating with $5$ developers.}

  \experience{2011---Present}%
  {Research Assistant}%
  {funded by}%
  {AFOSR}%
  {Dissertation research on computer vision models and algorithms for
    materials science image segmentation in \python, \numpy, \scipy,
    \opencv, and \matlab at the
    \institution{\href{http://cvl.cse.sc.edu/}{Computer Vision Lab}}
    at \institution{USC}.  Created a desktop GUI using
    \skill{wxWidgets} and a web interface using \django, \js, and
    \jquery. Conducted large-scale analysis using a 98-core
    high-performance computing system.}

  \experience{2011---Present}%
  {Project Manager}%
  {at}%
  {\href{http://palmettocomputerlabs.com/}{Palmetto Computer Labs}}%
  {Created and taught workshops on \git, the \linux command line,
    \android development, and open source software for hundreds
    of students, developers, and government officials at
    \institution{\href{http://it-ology.org/}{IT-oLogy}}.  Managed the
    \institution{\href{http://open-it-lab.com/}{Open IT Lab}} and
    associated projects. Assisted in planning
    \institution{\href{http://posscon.org/}{POSSCON}}.}

  \experience{2011}%
  {Contractor}%
  {for}%
  {Elastic Vision Consulting}%
  {Built a parser and generator for \skill{XML} medical records
    formats (CCR and CCD) in a \java web application.  Written
    using \skill{JDOM}, \skill{Xerces}, and \skill{Hibernate}, and run
    on an \skill{Axis2+Jetty6} driven server.}

  \experience{2010---2011}%
  {Research Assistant}%
  {for the}%
  {DARPA
    \href{http://www.darpa.mil/Our_Work/I2O/Programs/Minds_Eye.aspx}{Mind's
      Eye Program} }%
  {Researched video event recognition for the DARPA Mind's Eye
    program.  Collaborated with $10$ students and faculty members
    across three institutions.  Developed algorithms in
    \scheme, \bash, \matlab, and \c to
    process a corpus of 3480 videos extracted into over 1.5 million
    frames. Distributed processing over $7$ HPC machines.
    \href{http://0xab.com/research/video-in-sentences-out.html}{0xab.com/research/video-in-sentences-out.html}
    ,
    \href{https://www.github.com/malloc47/video-in-sentences-out}{github.com/malloc47/video-in-sentences-out}}

  \experience{2009---2010}%
  {NEH Fellow}%
  {at the}%
  {\href{http://cdh.sc.edu/}{USC Center for Digital Humanities}
    (\href{http://sapheos.org/}{Sapheos}/\href{http://cdh.sc.edu/paragon}{Paragon}
    Project)}%
  {Developed the prototype for a \emph{digital collation} application
    to identify sub-textual inconsistencies among multiple copies of
    \emph{The Faerie Queene} by \textsc{Edmund Spenser}.  Created in
    \matlab using \skill{VLFeat} and \opencv to process tens of
    thousands of book page images.
    \href{https://www.github.com/malloc47/digital-collation}{github.com/malloc47/digital-collation}}

  \experiencel{2007---2011}%
  {Teaching Assistant}%
  {for}%
  {\href{https://www.cse.sc.edu/}{USC Department of Computer Science
      and Engineering}}%
  {Taught classes in software development, web development, and
    computer engineering, covering \java, \js, \html, and
    \skill{Visual Basic}.  Created syllabi and course objectives,
    developed and graded projects and assignments, supervised labs,
    and tutored students.}

  % \experience{2005}%
  % {Intern---Technical Writer}%
  % {at}%
  % {JAARS, Inc.}%
  % {Created documentation and integrated context-sensitive online help
  %   system for speech and linguistic software written in C++ and
  %   Visual Basic.}

  % \experience{2001---2002}%
  % {Volunteer Software Developer}%
  % {at}%
  % {JAARS, Inc.}%
  % {Spearheaded the conversion from \skill{Visual Basic 4} to
  %   \skill{Visual Basic 6} for the linguistic reference tool
  %   \href{http://www.sil.org/computing/ipahelp/ipaprvw2.htm}{IPA
  %     Help}.}

\end{longtable}

\newcommand{\skills}[2]{
  \item #2 #1
}
\vspace{-0.5em}
\section{Skills \& Languages}
\vspace{-1em}
\begin{multicols}{4}
\raggedcolumns
\begin{itemize}
\renewcommand{\labelitemi}{}
\renewcommand{\skill}{\textnormal}
\setlength{\itemsep}{1pt}
\setlength{\parskip}{0pt}
\setlength{\parsep}{0pt}

%% \skills{Assembly}{\twonotes}
\skills{\bash}{\threeskill}
% \skills{Blender}{\twoskill}
\skills{\ccpp}{\threeskill }
\skills{Emacs Lisp}{\oneskill}
\skills{English}{\threeskill}
\skills{\git}{\twoskill}
\skills{GNU/\linux}{\threeskill}
\skills{\haskell}{\oneskill}
\skills{\html}{\threeskill}
%\skills{Processing}{\fivenotes Image}
\skills{\java}{\threeskill}
\skills{\js}{\twoskill}
\skills{\jquery}{\twoskill}
%\skills{LAMP Stack}{\fournotes}
\skills{\LaTeX}{\twoskill}
%% \skills{LISP}{\onenote}
%\skills{Learning}{\fournotes Machine}
% \skills{Maple}{\twoskill}
\skills{\matlab}{\threeskill}
%\skills{MS Office}{\fivenotes}
%\skills{Networking}{\threenotes}
\skills{\numpy/\scipy}{\threeskill}
\skills{\opencv}{\threeskill}
\skills{\php}{\oneskill}
\skills{\python}{\threeskill}
\skills{\django}{\twoskill}
\skills{\scheme}{\threeskill}
\skills{\sql}{\twoskill}
%% \skills{Sys. Admin.}{\threenotes}
%% \skills{Visual Basic}{\fivenotes}
%\skills{Windows}{\fivenotes}
%% \skills{Wordpress}{\fournotes}

\end{itemize}

\end{multicols}
\vspace{-1em}
  \begin{footnotesize}
    \oneskill Small-scale projects and/or assignments \hfill 
    \twoskill Multiple projects and/or experience teaching \hfill
    \threeskill Large-scale and/or multi-group projects
  \end{footnotesize}

\newcommand{\proj}[3]{
  \textsc{#1} & #2\\
   &\href{http://www.#3}{#3}\\
   \multicolumn{2}{c}{} \\ [-1ex]
}

\newcommand{\projl}[3]{
  \textsc{#1} & #2\\
   &\href{http://www.#3}{#3}\\
}

\newcommand{\projlh}[4]{
  \textsc{#1} & #2\\
   &\href{#3}{#4}\\
}
\section{Personal and Open Source Projects}
\begin{tabularx}{\textwidth}{@{}p{3cm}|X@{}}

  \proj{matsciseg}%
  {Framework for propagated 3D volume segmentation, used in my
    dissertation work.  Algorithms created in \python and \cpp and
    exposed as a web API using \django. Includes a web application
    that consumes the API created in \js, and \jquery.}%
  {github.com/malloc47/matsciseg}

  \proj{\href{http://nonpartisan.me}{nonpartisan.me}}%
{Google Chrome extension that filters social media websites for political keywords.  Available in the \href{https://chrome.google.com/webstore/detail/nonpartisanme/ninebcppidndhampaggnjbijpacoadgg}{Chrome Web Store}.  Featured in the \href{http://www.charlestoncitypaper.com/charleston/sick-of-politics-on-facebook-try-this-browser-tool/Content?oid=4153447}{Charleston City Paper}.}%
{github.com/malloc47/nonpartisan.me}

\proj{term-do}{An interactive terminal prompt that displays potential command completions as you type.  A hybrid of gnome-do and Emacs's ido-mode.  Works on many tested VT100 terminal types; built in~\cpp.  Includes client/server architecture implemented with boost.interprocess and full-featured plugin system. Available in the \href{https://aur.archlinux.org/packages/term-do-git/}{Arch Linux AUR}.}{github.com/malloc47/term-do}

  % \proj{Ratio Contour}{Maintainer and contributor for the Ratio Contour project, a salient object detection and segmentation method used for computer vision applications.  Developed in \c and \matlab.}{github.com/malloc47/ratio-contour}

  % \projl{PMLDAP}{\linux user management tool for Linux clusters.  Created as a simplified replacement for LDAP.  Capable of bootstrapping new systems, synchronizing users and configuration files, and running distributed commands.  Written in \bash.}{github.com/malloc47/pmldap}

  \projlh{Sina Weibo Mobile Client}{Created a \skill{J2ME}-based prototype mobile client for the popular Chinese \institution{Sina} microblogging service, similar to \institution{Twitter}.  Targeted at limited-functionality CLDC phones and uses a custom \java wrapper for the \institution{Sina} API.  Employs symmetric-key encryption for personal data.}{http://bd.weibo.10086.cn/2012/downloads_kjava}{bd.weibo.10086.cn/2012/downloads\_kjava}

  % \proj{Digital Collation}{Research prototype to ``collate'' high-resolution document scans using image registration.  Written in \matlab utilizing various computer vision libraries.}{www.github.com/malloc47/digital-collation}

  % \proj{matscicut}{An energy minimization framework for segmenting 3D materials volumes. Prototype of dissertation work, created in C++ using OpenCV libraries, with assorted MATLAB helper utilities.}{www.github.com/malloc47/matscicut}

  % \proj{git-hq}{A remote management system for git, coded in \python.}{www.github.com/malloc47/git-hq}

\end{tabularx}

%\section{Publications}
\let\originalbibitem\bibitem
\def\bibitem#1#2\par{%
  \noexpandarg
  \originalbibitem{#1}
  \StrSubstitute{#2}{Jarrell Waggoner}{\textbf{Jarrell Waggoner}}\par}

% \nocite{waggoner:13a}
% \nocite{waggoner:13b}
\nocite{waggoner:13c}
\nocite{waggoner:11}
\nocite{wang:11}
\nocite{temlyakov:10}
\nocite{zhang:10}
\nocite{waggoner:12}
\nocite{barbu:12}
\nocite{zhang:12}
% \nocite{temlyakov:13}
% \nocite{salvi:13a}
% \nocite{salvi:13b}

\renewcommand\refname{Selected Publications}
{\footnotesize \bibliography{cv}}
\bibliographystyle{plainyr-rev}

\section{Talks}
{\footnotesize
\begin{enumerate}[align=left,labelsep=0em]
\renewcommand{\labelenumi}{[\arabic{enumi}] }
\item \href{http://www.malloc47.com/posscon2013/}{Extending Django}.
  \emph{\href{http://posscon.org/}{Palmetto Open Source Software
      Conference}}.  Columbia, SC.  March 28, 2013.
\item \href{http://www.malloc47.com/cs-careers/}{Computer Science:
    Research, Industry, and Entrepreneurship}.  \emph{Careers in
    Science Lecture Series}.  Lancaster, SC.  March 6, 2013.
\item \href{http://www.malloc47.com/spie2013/}{Interactive Grain Image
    Segmentation Using Graph Cut Algorithms}.  \emph{SPIE
    (Computational Imaging XI)}.  Burlingame, CA.  February 6, 2013.
\item Android Application Development Workshop.  \emph{Appathon
    Contest}.  Columbia, SC.  Nov. 17, 2012.
\item Open Source and Education. \emph{SC Municipal Technology
    Association (SCMTA) Conference}. Charleston, SC.  Sep. 6, 2012.
\item Open Source and Higher Education.  \emph{SC Technical College
    System (SCTCS) Conference}.  Columbia, SC.  Sep. 25, 2012.
\item Introduction to Android Development.  \emph{Digital Humanities
    High Performance Computing (DHHPC) Workshop}.  Columbia, SC.  Aug.
  8, 2012.
\item Combining Global Labeling and Local Relabeling for
  Metallic Image Segmentation.  \emph{SPIE (Computational Imaging X)}.
  Jan. 23, 2012.
\item Open Source and Government.  \emph{SC Government Management
    Information Systems (SCGMIS) Workshop.}  Columbia, SC.  Jan. 19,
  2012.
\end{enumerate} }

\section{Honors/Awards at USC}
\begin{tabularx}{\textwidth}{@{}r|X l|p{4.9cm}@{}}
2012 & \small{Gamecock Computing Research Symposium Poster Session,  First Place} &
2004 & Clara P. Hammond Award  \\

2012 & Graduate Student Day Presentation,  First Place &
2004 & Science and Mathematics Award \\

2009 & Upsilon Pi Epsilon &
2004 & Highest Academic Average Award \\
\end{tabularx}

\section{Activities}
teaching, programming, open source software, system administration,
data visualization, Linux,
\href{https://soundcloud.com/malloc47}{music composition}

\null\vfill
\footnotesize{
  Online:  \href{http://resume.malloc47.com}{resume.malloc47.com} \hfill
  Full CV: \href{http://cv.malloc47.com}{cv.malloc47.com} \hfill 
  Source:  \href{https://github.com/malloc47/cv}{github.com/malloc47/cv/}
}

\pagestyle{myheadings}
\markright{Jarrell Waggoner}

%%\XeTeXpdffile ''resume.pdf'' page 1 scaled 800

\end{document}
