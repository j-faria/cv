\newif\ifoutreach%
% \outreachtrue%              comment out to hide outreach

\newif\ifskills%
%\skillstrue%               comment out to hide skills section

\newif\iftalksposters%
\talksposterstrue%        comment out to hide talks and posters section

\newif\ifconferences%
% \conferencestrue%         comment out to hide conferences section

\newif\ifreferees%
% \refereestrue%            comment out to hide referees

\newif\ifinterests%
\intereststrue%             comment out to hide interests

\newif\ifoutcomes%
\outcomestrue%              comment out to hide research outcomes

\newif\ifpaperlistattached
% \paperlistattachedtrue%   comment out to hide mention to attached list of papers

\newif\ifallpapers
% \allpaperstrue%           comment out to hide full list of papers

\newif\ifcoauthor%
% \coauthortrue%            comment out to hide co-authors

\newif\ifteaching
\teachingtrue%              comment out to hide teaching experience 

\newif\iftechnicalskillslanguages
% \technicalskillslanguagestrue% comment out to hide technical skills and languages

\documentclass[10pt]{article}

\usepackage{marvosym}
\usepackage{wasysym}
\usepackage{fontspec}
\usepackage{xunicode,xltxtra,url,parskip}
\defaultfontfeatures{Scale=MatchLowercase,Mapping=tex-text}
\RequirePackage{color,graphicx}
\usepackage[usenames,dvipsnames]{xcolor}
\usepackage[left=1in, right=1in, top=0.8in, bottom=1in,showframe=false]{geometry}
\usepackage{supertabular}
\usepackage{soul}
\usepackage{titlesec}
\usepackage{multicol}
\usepackage{multirow}
\usepackage{longtable}
\usepackage{xstring}
\usepackage{rotating}
\usepackage{ifthen}


\usepackage[xetex,
            unicode,
            pdfencoding=auto,
            pdfinfo={
              Title={João Faria / CV},
              Author={João Faria},
              Subject={João Faria / CV},
              Keywords={astrophysics},
              Producer={xelatex},
              Creator{xelatex}
            },
            ]{hyperref}
\newcommand{\MYhref}[3][blue]{\href{#2}{\color{#1}{#3}}}%


%%%%%%%%%%%%%%%%%%%%%
% headers, footers
%%%%%%%%%%%%%%%%%%%%%
\usepackage{lastpage}
\usepackage{fancyhdr}
\fancypagestyle{default}{
  \fancyhf{}% No header/footer
  \renewcommand{\headrulewidth}{0pt}% No header rule
  % \fancyfoot[L]{}
  % \fancyfoot[R]{\footnotesize \thepage/\pageref*{LastPage}}% Right footer
  \fancyfoot[R]{\footnotesize João P. \textsc{Faria} {\tiny \opposition}  
                              \href{mailto:***email**}{***email**} {\tiny \conjunction}  
                              \thepage/\pageref*{LastPage}}% Right footer
}
%
% footer for the last page with extra info
\newcommand{\footerlastpage}{%
\footnotesize{
  Last updated: \today %\quad---\quad
  % Online:  \href{http://j-faria.github.io/cv}{\textasciitilde jfaria/cv} \quad---\quad
  % Résumé: \href{http://resume.malloc47.com}{resume.malloc47.com} \hfill
  % online: \href{https://github.com/j-faria/cv}{github.com/j-faria/cv/}
}
}
%
% footer (and possibly header if signed) for the first page
\fancypagestyle{firststyle}{
   \fancyhf{}
   ***signature**
   %\fancyhead[R]{\today \\ \includegraphics[height=1.5\baselineskip]{signature}}
   \fancyfoot[R]{\footnotesize\thepage/\pageref*{LastPage}}% Right footer
}
%
%
\fancypagestyle{last-page}{
  \fancyhf{}% No header/footer
  \renewcommand{\headrulewidth}{0pt}% No header rule
  \fancyfoot[L]{\footnotesize \footerlastpage}
  \fancyfoot[R]{\footnotesize\thepage/\pageref*{LastPage}}% Right footer
}

\fancypagestyle{last-page-no-number}{
  \fancyhf{}% No header/footer
  \renewcommand{\headrulewidth}{0pt}% No header rule
  \fancyfoot[L]{\footnotesize \footerlastpage}
  % \fancyfoot[R]{\footnotesize\thepage/\pageref*{LastPage}}% Right footer
}


***PageStyle1**

%%% this should be 
% \pagestyle{default}
% \thispagestyle{firststyle}
%%% and don't forget PageStyle2





\usepackage[absolute]{textpos}
\usepackage{enumitem}
\usepackage{tabularx}
\usepackage{type1cm}
\usepackage{lettrine}

\usepackage{xpatch}
\usepackage[backend=bibtex,
            style=authoryear-icomp,
            % style=numeric
            citestyle=authoryear,
            maxbibnames=1,
            % sorting=ydnt,
            sorting=none,
           ]{biblatex}
\addbibresource{cv.bib}

\DefineBibliographyStrings{english}{%
  andothers = {et\addabbrvspace al\adddot}
  % andothers = {+}
}

% remove quotes around the title of the paper
\DeclareFieldFormat[article, inbook, incollection, inproceedings, misc, thesis, unpublished]
{title}{#1}

% for some types replace "pages" by "p."
\DeclareFieldFormat[inproceedings, incollection, inbook]
{pages}{p. #1}

\renewbibmacro{in:}{}
\AtEveryBibitem{\clearfield{issn}}
\AtEveryBibitem{\clearfield{urldate}}
\AtEveryBibitem{\clearfield{month}}
\AtEveryBibitem{\clearfield{eprint}}
\AtEveryBibitem{\clearfield{eid}}
\DeclareFieldFormat{pages}{#1}
\ExecuteBibliographyOptions{doi=false}
\DeclareFieldFormat{doilink}{%
\iffieldundef{doi}{#1}%
{\href{http://dx.doi.org/\thefield{doi}}{#1}}}


\DeclareBibliographyDriver{article}{%
  \usebibmacro{bibindex}%
  \usebibmacro{begentry}%
  \usebibmacro{author/translator+others}%
  \addcomma\space\newblock
  \printtext[doilink]{%
  \usebibmacro{journal+issuetitle}%
  % \newunit
  % \usebibmacro{byeditor+others}%
  \newunit
  \usebibmacro{note+pages}%
  }%
  % \newunit\newblock
  % \iftoggle{bbx:isbn}
  %   {\printfield{issn}}
  %   {}%
  \newunit\newblock
  \usebibmacro{doi+eprint+url}%
  \newunit\newblock
  \usebibmacro{addendum+pubstate}%
  \setunit{\bibpagerefpunct}\newblock
  \newblock\newline
  \usebibmacro{title}%
  \usebibmacro{pageref}%
  \usebibmacro{finentry}%
}

\input{boldauthorname}
\boldname{Faria}{Jo}{J.\bibnamedelima P.}


\newcommand{\mnras}{MNRAS}
\newcommand{\aap}{A\&A}
\newcommand{\aj}{AJ}
\newcommand{\apj}{ApJ}



% \usepackage{natbib}
% \usepackage{multibbl}
% \usepackage[authoryear]{natbib}
% \usepackage[style=authoryear,natbib=true]{biblatex}
% \addbibresource{cv.bib}

% \makeatletter
% \renewcommand*{\@biblabel}[1]{\hfill[A#1]}
% \makeatother

% \usepackage{multibib}
% \newcites{first}{First author}
% \newcites{co}{Co-author}

\usepackage{etaremune}
\makeatletter
\long\def\thebibliography#1{%
  \section*{\refname}%
  \@mkboth{\MakeUppercase\refname}{\MakeUppercase\refname}
  \settowidth{\dimen0}{\@biblabel{#1}}%
  % \setlength{\dimen2}{\dimen0}%
  \setlength{\dimen2}{0pt}%
  \addtolength{\dimen2}{\labelsep}
  \sloppy
  \clubpenalty 4000 
  \@clubpenalty 
  \clubpenalty 
  \widowpenalty 4000
  \sfcode `\.\@m
  % \renewcommand{\labelenumi}{\@biblabel{A\theenumi}\label{A\theenumi}} % labels like [A3], [A2], [A1]
  \renewcommand{\labelenumi}{-\label{A\theenumi}}
  \begin{etaremune}[labelwidth=\dimen0,leftmargin=\dimen2]\@openbib@code
}
\def\endthebibliography{\end{etaremune}}
\def\@bibitem#1{%
  \item \if@filesw\immediate\write\@auxout{\string\bibcite{#1}{\the\value{enumi}}}\fi\ignorespaces
}
\makeatother


\newcommand{\myprenote}{%
\begin{center}
  \begin{center}
  \begin{tabular}{ r l }
  Full list in
  \href{https://ui.adsabs.harvard.edu/\#/public-libraries/OtQQjddpThGyGAflYLVxng}{ADS}%
  & 
  total_papers_and_citations \\
  % \textbf{84} refereed papers. \textbf{2769} citations\\
  %
  & first_author_papers_and_citations
  % & \textbf{100} first-author papers. \textbf{100} citations
  \end{tabular}
  \end{center}
  % \begin{table}
  
  % &
  % --
  % \end{table}
  % \hskip-.5em or at \href{https://joaofaria.space/publications}{joaofaria.space/publications} \qquad % \hspace{4cm} 
% \textbf{4} self-citations
%  \quad \textbf{total refereed citations:} 206 \\ 
%  \textbf{metrics:} H=10;~~m=1.4;~~g=15;~~i10=10;~~tori=0.5;~~riq=101;~~read10=312.9
\end{center}
\vskip1em
}


\include{functions}

\begin{document}
% \pagestyle{empty}

\par{\centering {\Huge João P. \textsc{Faria} }\bigskip\par}
% \par{\centering {\large curriculum vitae}\bigskip\par}

\begin{multicols}{2}
\setlength{\parskip}{0pt}

\section{Contact}
\begin{tabularx}{\linewidth}{@{}l X@{}}
  \textsc{Address} & \small{***institute**} \\
                   & \small{***address**} \\
  \textsc{Phone}   & \small{***phone**}\\
  \textsc{email}   & \href{mailto:***email**}{***email**} \\
\end{tabularx}

% \vfill
\columnbreak

\section{Online}
\begin{tabularx}{\linewidth}{@{}l X@{}}
  \textsc{Website}  & \href{***websiteescaped**}{***website**} \\
  \textsc{github}   & \href{http://www.github.com/***github**}{github.com/***github**}\\
  % \textsc{Twitter}  & \href{https://twitter.com/***twitter**}{@***twitter**} \\
  \textsc{Orcid}    & \href{http://orcid.org/***orcid**}{***orcid**} \\
  % \textsc{LinkedIn}    & \href{http://www.linkedin.com/in/}{linkedin.com/} \\
  \vspace{\fill}
\end{tabularx}

\end{multicols}


\ifinterests

  \section{Research Interests}

  ***interests**

\fi

\newcommand{\degree}[7]{\textsc{#1} & #2 in \textsc{#3} & \textbf{#4}\\
                        & \ifthenelse{\equal{#5}{}}{\small ~}{\small Advisors: #5} &
                          \ifthenelse{\equal{#7}{}}{\small ~}{\small #7}\\
                        & \ifthenelse{\equal{#6}{}}{~}{\footnotesize Dissertation: ``#6''} & \\[1ex]
                        }

\newcommand{\experience}[4]{#1 & #2 \\
                            % if there is a third argument, print it
                            \ifthenelse{\equal{#3}{}}{}{ &\emph{#3}\\ }
                            \ifthenelse{\equal{#4}{}}{}{ &\footnotesize{#4} \\ }
                            % \multicolumn{2}{c}{} \\ [-1ex]
                            }

\section{Academic background}
  \begin{tabularx}{\linewidth}{@{}r X l}
    % &\normalsize \textsc{GPA}: 3.8/4.0 | \small\emph{magna cum laude} \\[1ex]%

    ***degrees**

  \end{tabularx}

  \vspace{-3em}


\section{Research Experience}
\begin{tabularx}{\linewidth}{@{}r X l}

  \experience{2023 - present}
             {Research \& Teaching Fellow}
             {Department of Astronomy, University of Geneva, Switzerland}
             {}

  \experience{2018 - 2023}
             {Post-doctoral Researcher}
             {Institute of Astrophysics and Space Sciences, Porto, Portugal}
             {}

  % \experience{2019 - future}
  %            {Researcher}
  %            {where?}
  %            {}
\end{tabularx}


%%%% PUBLICATIONS !

\begin{refsection}
  % \nocite{*}
  \nocite{Faria2025, Faria2022, Faria2020, Faria2018, Faria2016, Faria2016a}
  % \boldname{Faria}{João}{J. P.}
  \section{Publications}
  % \defbibheading{subbibliography}[\refname]{\textbf{#1}}
  \myprenote
  \printbibliography[title=Selected first-author publications, heading=subbibliography]
\end{refsection}
\begin{refsection}
  \defbibheading{subbibliography}[\refname]{\vskip1ex\textbf{#1}}
  
  \ifallpapers % list ALL the papers!
    \nocite{%
      % Suarez-Mascareno2020,
      Nelson2020,
      Pereira2019,
      Sousa2019,
      Oshagh2018,
      Martins2018,
      Serrano2018,
      Ulmer-Moll2019,
      Lillo-Box2018,
      Santerne2018,
      Barros2018,
      Santos2018,
      Santos2017a,
      Lillo-Box2018a,
      Oshagh2017,
      Barros2017,
      Santos2017,
      Adibekyan2016a,
      Santos2016,
      Figueira2016a,
      Adibekyan2016,
      Figueira2016,
      Adibekyan2015,
      Martins2015,
      Mortier2015,
      Figueira2014,
      Verma2014,
      Santos2014,
      Faria2012}
    
    \printbibliography[title=Other selected publications, heading=subbibliography]
  
  \else % list only selected papers
    \ifcoauthor
      \nocite{Santos2018,Santos2016,Figueira2016a,Mortier2015a,Mortier2015b,%
              Santos2014,Oshagh2017,%Lillo-Box2017,%Barros2017,
              }
      \ifpaperlistattached
        \printbibliography[title=Other selected publications (full list in accompanying document), %
        heading=subbibliography]
      \else
        \printbibliography[title=Other selected publications, heading=subbibliography]
      \fi
    \fi
  \fi

\end{refsection}

%% old way with simple bibliography
% \let\originalbibitem\bibitem
% \def\bibitem#1#2\par{%
%   \noexpandarg
%   \originalbibitem{#1}
%   \StrSubstitute{#2}{J.~P. {Faria}}{\textbf{J.~P. Faria}}\par}

% \nocite{*}
% \renewcommand\refname{Publications}
% \bibliographystyle{plainyr-rev}
% \bibliography{cv}
%% to work with multibbl
% \bibliography{first}{cv}{First-author}


\ifteaching


\renewcommand{\experience}[4]{\textsc{#1} & #2 \\
                            % if there is a third argument, print it
                            \ifthenelse{\equal{#3}{}}%
                              {}%
                              { & \emph{#3}\\}
                            &\footnotesize{#4} \\
                            \multicolumn{2}{c}{} \\ [-1ex]
                            }

% \newpage
% \vspace{2em}

\section{Teaching Experience}
\vspace{-1em}
\begin{longtable}{r|p{11cm}}

\experience{2019, 2021, 2023}%
{Exoplanets: Detection, Characterization, Population}
{Doctoral Program in Astronomy, University of Porto}
{Course for first-year PhD students; 15h of lectures}

\experience{July 17-27, 2016}%
{Detecting Planetary Transits and Radial-Velocity Signals}
{IVth Azores International Advanced School in Space Sciences}
{4.5hrs tutorial in the Summer school ``Asteroseismology and Exoplanets: Listening to the Stars\newline and Searching for New Worlds'' covering data analysis methods for the detection of exoplanets \newline in transit lightcurves and radial-velocity datasets.
DOI: \href{http://www.springer.com/gp/book/9783319593142}{10.1007/978-3-319-59315-9}}

\experience{February, 2016}%
{Bayesian Statistics}%
{Advanced Course, Institute of Astrophysics and Space Sciences}%
{Created and lectured a course about Bayesian statistics, composed of a seminar talk (2h)
 \newline and two hands-on computational classes (6h). Lecture notes are also available.
 \href{https://github.com/iastro-pt/Bayes-IA}{Online repository}.}


\experience{November, 2015}%
{Python for Astronomers}%
{Advanced Course, Center for Astrophysics University of Porto}%
{Teaching assistant for introductory programming classes.
 \href{https://github.com/iastro-pt/python-for-astronomers}{Online repository}.}


\experience{Aug 31 - Sep 4, 2015}%
{Escola de Verão de Física (Physics Summer School)}%
{Summer School, Faculty of Sciences, University of Porto}%
{Supervised and lectured a project about ``The energy of stars'' for a class of 5 high-school students.}

\experience{March 26-28, 2014}%
{Stellar evolution models with the MESA code}%
{Advanced Course, Center for Astrophysics University of Porto}%
{Developed and coordinated a series of three hands-on tutorial classes 
demonstrating the use\newline of the~\texttt{MESA} stellar evolution code 
(Paxton et al. \href{http://dx.doi.org/10.1088/0067-0049/192/1/3}{2011};
               \href{http://dx.doi.org/10.1088/0067-0049/208/1/4}{2013}).}

\end{longtable}

\fi


%%% RESEARCH
\ifoutcomes

\section{Other research outcomes}

\begin{itemize}\itemsep-0.3em
  \item PoET Science Team and ESPRESSO Science Team: Working Group leader
  \item PI of FCT exploratory project (EXPL/FIS-AST/0615/2021, SAM); awarded 45 k€
  % \item ESPRESSO Science Team \textit{collaborator} (analysis of RV data, selection of targets)
  \item PI of cooperation project FCT-DAAD (Porto-Göttingen); awarded 2000€
  \item PI of \textbf{4} ESO observing proposals (ESPRESSO\,\raisebox{0.8\depth}{@}\,VLT),
        co-I of \textbf{20} accepted ESO/OPTICON observing proposals
  % \item SPIRou legacy program \textit{additional collaborator}
  \item co-I of \textbf{2} accepted ISSI International Team proposals
  \item member of the PLATO 
  \href{https://warwick.ac.uk/fac/sci/physics/research/astro/plato-science/research/researchareas/exoplanets/wp115100}{Work Package 115100},
  dealing with \emph{Astrophysical Noise Sources}
  \item observing experience from several missions to the La Silla and Paranal observatories
  \item referee for A\&A, MNRAS, ApJ, and AJ

\end{itemize}

\fi




% \newpage
% \vspace{-1em}

%%% OUTREACH !
\ifoutreach

  \section{Outreach}
    \begin{itemize}
    % \renewcommand{\labelenumi}{[G\arabic{enumi}] }
    % \item \emph{AstroPOD}, a podcast where children ask questions and astronomers give answers
    %       (2020, conception)\\[-1.5em]
    \item \emph{Press coverage} on the detection of Proxima \emph{d}
          (\href{https://doi.org/10.1051/0004-6361/202142337}{Faria+ 2022}):
          \href{https://www.eso.org/public/news/eso2202/}{ESO PR}, 
          \href{https://divulgacao.iastro.pt/pt/2022/02/10/proxima-d/}{National PR}
    \item \emph{Press coverage} on the first ESPRESSO results (\href{https://arxiv.org/abs/1911.11714}{Faria+$\,$2019}):
          \href{http://www.iastro.pt/news/news.html?ID=123}{PR},
          several news stories 
            [\href{https://zap.aeiou.pt/constelacao-do-pintor-hipotetico-sistema-planetario-afinal-atividade-da-estrela-298038}{1},
            \href{https://www.jn.pt/mundo/hipotetico-sistema-planetario-e-afinal-atividade-da-estrela-dizem-investigadores-11632495}{2}%
            ],
          \href{http://divulgacao.iastro.pt/pt/feature/observacao-de-planetas-extrassolares/}{radio interview}. \\[-1.5em]
    % http://divulgacao.iastro.pt/pt/feature/observacao-de-planetas-extrassolares/
    \item \emph{CoAstro}, a project to involve primary school teachers in research projects, 2018 
          (\href{http://condominio.astro.up.pt/}{more info, in Portuguese}).\\[-1.5em]
    \item \emph{Ignite IAstro (multiple sessions)}, short outreach presentations in small towns %
          (\href{http://www.iastro.pt/outreach/activities/ignite/}{more info, in Portuguese}).\\[-1.5em]
    \item \emph{Mentes que Brilham (TV show)}, interview about the work published in \href{http://adsabs.harvard.edu/cgi-bin/nph-abs_connect?fforward=http://dx.doi.org/10.1051/0004-6361/201425298}{Martins+ (2015)}, May 13, 2015. \\[-1.5em]
    \item \emph{Universidade Júnior (University of Porto)}, monitor of scientific activities for middle-school students, July 2013. \\[-1.5em]
    \item \emph{Planetarium sessions (presenter)}, as collaborator of the Outreach Unit of CAUP, Summer 2013. \\[-1.5em]
    \item \emph{Noites no Observatório}, monthly outreach events at the Lisbon Astronomical Observatory, 2010-2011.
    \end{itemize}

\fi
%%% END OUTREACH !


% \vskip1em

%%% SUPERVISION and SOFTWARE !
% \newpage

\vskip\baselineskip
\begin{multicols}{2}
\section{Supervision}
~\vskip-1em
\section{Scientific software}
\vskip-1em
\end{multicols}
\vspace{-3em}
\begin{multicols}{2}\begin{itemize}[leftmargin=1em]
  \item[-] Ph.D. (co-supervisor): Eduardo Gonçalves\\
           \emph{``Leveraging Sun-as-a-star high-resolution spectra [...]''}\\[-1.2em]
  \item[-] Ph.D. (co-supervisor): João D. Camacho\\
           \emph{``Statistical Analysis for the Detection of Other Earths''}\\[-1.2em]
  \item[-] M.Sc.-level fellowship: João Gomes da Silva \\
           \emph{new activity indicators; development of \href{https://github.com/gomesdasilva/ACTIN}{ACTIN}}\\[-1.2em]
  \item[-] M.Sc.: João D. Camacho\\
           \emph{``Gaussian processes [... for ...] exoplanet search''}\\[-1.2em]
  \item[-] Undergraduate projects: Isham Mishra, Ana Barboza, Paul Charpentier,
           Sofia Iñiguez, among others \vfill\null
  \columnbreak
  % \item[]\\
  % \item[]
% \end{itemize}
% \begin{itemize}[leftmargin=*]
  \item[-] \href{https://www.kima.science}{kima}: %
          Exoplanet detection in RVs with DNest4 and GPs, 
          described in \href{http://dx.doi.org/10.21105/joss.00487}{Faria et al. (2018)} \\[-1.6em]
  \item[-] \href{http://github.com/j-faria/arvi}{arvi}: %
          Tools to download and analyse RV data\\[-1.6em]
  \item[-] \href{http://github.com/j-faria/iCCF}{iCCF}: %
          Tools for analysis of CCFs and activity indicators \\[-1.6em]
  \item[-] \href{http://github.com/j-faria/bgls}{BGLS}: %
           Bayesian version of the Generalized Lomb-Scargle periodogram, described in 
           \href{http://dx.doi.org/10.1051/0004-6361/201424908}{Mortier et al. (2015)}  \\[-1.6em]
  \item[-] \href{http://github.com/j-faria/kumaraswamy}{kumaraswamy} 
           and \href{http://github.com/j-faria/loguniform}{loguniform}: %
           Implementation of common probability distributions \\[-1.6em]
  % \item[-] \href{http://github.com/j-faria/OPEN}{OPEN}: %
  %          A platform for exoplanet detection in RVs  \\[-1.6em]
  % \item[-] \href{https://bitbucket.org/pedrofigueira/bayesiancorrelation}{pfigueira/BayesianCorrelation} (contribution)\\
\end{itemize}
\end{multicols}



\iftechnicalskillslanguages
  %%% TECHNICAL SKILLS AND LANGUAGES !
  % \newpage

  \vskip\baselineskip
  \begin{multicols}{2}
  \section{Technical skills}
  ~\vskip-1em
  \section{Languages}
  \vskip-1em
  \end{multicols}
  \vspace{-3em}
  \begin{multicols}{2}\begin{itemize}[leftmargin=1em]
    \item[-] Programming: Python, C\texttt{++}/C, Fortran, R, Shell \\[-1.6em]
    \item[-] Markup: HTML (+CSS+JS), \LaTeX, Markdown
    \vfill\null
    \columnbreak
  % \end{itemize}
  % \begin{itemize}[leftmargin=*]
    \item[-] Portuguese (Native)  \\[-1.6em]
    \item[-] English (Proficient) \\[-1.6em]
    \item[-] French (A2 level) \\[-1.6em]
    \item[-] Spanish (Beginner)\\ 
  \end{itemize}
  \end{multicols}
\fi

%%% END OF TECHNICAL SKILL AND LANGUAGES


% \vspace{2em}

\ifskills

  \begin{minipage}{\linewidth}
    \newcommand{\skills}[2]{
      \item #2 #1
    }
    \section{Skills \& Languages}
      \begin{multicols}{4}
        \raggedcolumns
        \begin{itemize}
          \renewcommand{\labelitemi}{}
          \renewcommand{\skill}{\textnormal}
          \setlength{\itemsep}{1pt}
          \setlength{\parskip}{0pt}
          \setlength{\parsep}{0pt}

          % \skills{\bash}{\oneskill}
          \skills{\python}{\threeskill}
          \skills{\ccpp}{\twoskill}
          %% \skills{English}{\threeskill}
          \skills{\git}{\threeskill}
          \skills{Fortran}{\threeskill}
          \skills{GNU/\linux}{\threeskill}
          %% \skills{\html}{\threeskill}
          \skills{\LaTeX}{\twoskill}
          %\skills{Learning}{\fournotes Machine}
          \skills{\matlab}{\oneskill}

        \end{itemize}
      \end{multicols}
      \begin{footnotesize}
        \raggedleft
          Large-scale and/or multi-group projects ~ \threeskill \\
          Multiple projects and/or experience teaching \twoskill \\
          Small projects or assignments \oneskill \\
      \end{footnotesize}

      \begin{center}
      \noindent\rule{8cm}{0.4pt}
      \end{center}

      \begin{multicols}{4}
        \begin{itemize}
        \renewcommand{\labelitemi}{}
        \setlength{\itemsep}{1pt}
        \setlength{\parskip}{0pt}
        \setlength{\parsep}{0pt}

        \skills{Portuguese}{\threeskill}
        \skills{French}{\oneskill}
        \skills{English}{\twoskill}
        \skills{Spanish}{\oneskill}
        \skills{~}{}
        \skills{~}{}
        \end{itemize}
      \end{multicols}
      \begin{minipage}{0.5\linewidth}
        \begin{footnotesize}
        \raggedleft
           Native speaker \threeskill \\
          Proficient \twoskill \\
          Basic understanding \oneskill \\
        \end{footnotesize}
      \end{minipage}

  \end{minipage}

\fi


\iftalksposters
%%% TALKS and POSTERS

% \newpage
\section{Talks / Posters}

\textbf{Invited}
\begin{enumerate}

  ***invited**

\end{enumerate}

\textbf{Contributed} (abridged list; T: talk, P: poster, S: splinter session)
\renewcommand{\labelenumi}{[P\arabic{enumi}] }
\begin{enumerate}\itemsep-0.3em

  ***postersANDtalks**

\end{enumerate}

\fi


\ifconferences

  \section{Participation in Conferences}
  \begin{itemize}

    ***conferences**

  \end{itemize}

\fi


\ifreferees

  % \vspace{3em}
  \vspace{5cm}
  % \null
  % \vfill
  % \vspace{\fill}%vfill
  {\Large\scshape\raggedright\sffamily Referees}
  \newcommand{\service}[2]{
    \textsc{#1} & #2\\[1em]
    %% \multicolumn{2}{c}{} \\
  }
  \newcommand{\mail}[1]{
    \href{mailto:#1}{#1}
    %% \multicolumn{2}{c}{} \\
  }
  \begin{longtable}{r|p{10cm}}
    \service{Nuno C. Santos}{\mail{nuno.santos@astro.up.pt}\newline
                             Institute of Astrophysics and Space Sciences, Porto, Portugal}
    \service{Pedro Figueira}{\mail{pedro.figueira@astro.up.pt}\newline
                             Institute of Astrophysics and Space Sciences, Porto, Portugal}
    \service{Xavier Dumusque}{\mail{xavier.dumusque@unige.ch}\newline
                             Observatoire de Genève, Switzerland}
  \end{longtable}

\fi

% \AtEndDocument{%
%   % end-of-document content
% % \null\vfill
% \vskip2em
% \footnotesize{
%   Last updated: \today \quad---\quad
%   Online:  \href{http://j-faria.github.io/cv}{\textasciitilde jfaria/cv} \quad---\quad
%   % Résumé: \href{http://resume.malloc47.com}{resume.malloc47.com} \hfill
%   Source:  \href{https://github.com/j-faria/cv}{github.com/j-faria/cv/}
% }
% %
% }

***PageStyle2**
%%% this should be
% \thispagestyle{last-page}

\end{document}






%\section{Travel}
%\begin{enumerate}
%\renewcommand{\labelenumi}{[T\arabic{enumi}] }
%\item \emph{Mind's Eye PI Meeting}. DARPA Project P.I.~meeting. Denver, CO. January 20---21, 2011.
%\item \emph{Tomography and its Applications to Materials Science and Non-Destructive Evaluation}. Organizd by M. De Graef, L. Drummy, J. Simmons, M. Comer, C. Bouman, and J. Knopp. Tech\^{}Edge, Dayton, Ohio. December 13---15, 2010.
%\item Visiting scholar. In collaboration with J. M. Siskind. Purdue University, West Lafayette, IN. December 6---22, 2010 \& January 5---16, 2011.
%\item \emph{Mind's Eye Kickoff Meeting}. DARPA Project P.I.~meeting. San Diego, CA. September 23---24, 2010.
%\end{enumerate}

% \section{Honors/Awards}
% \begin{center}
% \begin{tabular*}{0.75\textwidth}{rll}
% 2012 & Gamecock Computing Research Symposium Poster Session,  First Place & \\
%      & Graduate Student Day Presentation,  First Place \\
% 2011 & Graduate Student Day Presentation,  Second Place \\
% 2010 & Graduate Student Day Presentation,  Honorable Mention \\
% 2009 & Upsilon Pi Epsilon \\
% % \multicolumn{3}{r}{}\\
% % 2006 & Senior Computer Science Award & {\lighttext \textcolor{lightg}{Bryan College}}\\
% \multicolumn{3}{r}{}\\
% 2004 & Clara P. Hammond Award & \\
% & Science and Mathematics Award \\
% & Highest Academic Average Award \\
% \end{tabular*}
% \end{center}

%\section{Professional Societies}

% \section{Classes Taught}
% \vspace{-1em}
% \begin{center}
% %\begin{tabular*}{0.75\textwidth}{r @{\hspace{0.5em}\textcolor{lightg}{\symbol{"00BB}}\hspace{0.5em}} l @{\extracolsep{\fill}} l }
% \begin{tabular*}{0.75\textwidth}{r @{\hspace{0.5em}\textcolor{lightg}{\symbol{"00BB}}\hspace{0.5em}} l l c }
% %\multicolumn{3}{r}{University of South Carolina}\\
% 2012--2013 & Open Source 101 & Open Source Software & \\
% 2012--2013 & Version Control 101 & git, github \\
% 2012--2013 & Command Line 101 & Linux, BASH \\
% \multicolumn{3}{r}{}\\
% Fall 2011 & CSCE 211 & Digital Logic Design &  \\
% Summer II 2008 & CSCE 102 & HTML/CSS/JavaScript \\
% Spring 2008 & CSCE 145 Lab & Java \\
% Fall 2007 & CSCE 145 Lab & Java \\
% \multicolumn{3}{r}{}\\
% %\multicolumn{3}{r}{University of South Carolina at Lancaster}\\ \hline
% Spring 2007 & CSCE 204 & Visual Basic & \\
% Spring 2007 & Math 241 \& Math 242 & Maple \\
% \end{tabular*}
% \end{center}

% \vspace{2em}

% \section{Service}
% \vspace{-1em}
% \newcommand{\service}[2]{
%   \textsc{#1} & #2\\
%   %% \multicolumn{2}{c}{} \\
% }
% \begin{longtable}{r|p{10cm}}
%   \service{Itern mentoring}{Groupon internship program, 2014}
%   \service{Webmaster}{\href{http://cvl.cse.sc.edu/wvm2013/}{Winter Vision Meetings, 2013}}
%   \service{Webmaster}{\href{http://cvl.cse.sc.edu/wacv2013/}{Workshop on the Applications of Computer Vision, 2013}}
%   \service{Judge}{Discovery Day --- Undergraduate Research Presentations}
%   \service{Reviewer}{Pattern Recognition Letters}
%   \service{Reviewer}{IEEE Transactions on Pattern Analysis and Machine Intelligence}
%   %% \service{Fellow}{NSF GK-12 Program}
%   \service{Member}{Institute of Electrical and Electronics Engineers (IEEE)}
%   \service{SysAdmin}{Computer Vision Lab}
% \end{longtable}

% \vspace{1em}

% \section{Personal and Open Source Projects}
%   \newcommand{\proj}[3]{
%     \textsc{#1} & #2\\
%      &\href{http://www.#3}{#3}\\
%      \multicolumn{2}{c}{} \\ [-1.5ex]
%   }

%   \newcommand{\projlh}[4]{
%     \textsc{#1} & #2\\
%      &\href{#3}{#4}\\
%   }

%   \begin{longtable}{@{}p{3cm}|p{13cm}}

%     \proj{matsciseg}%
%     {Framework for propagated 3D volume segmentation, used in my dissertation work.  Algorithms created in \python and \cpp and exposed as a web API using \django. Includes a web application that consumes the API created in \js, and \jquery.}%
%     {github.com/malloc47/matsciseg}

%   %   \proj{\href{http://nonpartisan.me}{nonpartisan.me}}%
%   % {Google Chrome extension that filters social media websites for political keywords.  Available in the \href{https://chrome.google.com/webstore/detail/nonpartisanme/ninebcppidndhampaggnjbijpacoadgg}{Chrome Web Store}.  Featured in the \href{http://www.charlestoncitypaper.com/charleston/sick-of-politics-on-facebook-try-this-browser-tool/Content?oid=4153447}{Charleston City Paper}.}%
%   % {github.com/malloc47/nonpartisan.me}

%   % \proj{term-do}{An interactive terminal prompt that displays potential command completions as you type.  A hybrid of gnome-do and Emacs's ido-mode.  Works on many tested VT100 terminal types; built in~\skill{C++}.  Includes client/server architecture implemented with boost.interprocess and full-featured plugin system. Available in the \href{https://aur.archlinux.org/packages/term-do-git/}{Arch Linux AUR}.}{github.com/malloc47/term-do}

%     % \proj{Ratio Contour}{Maintainer and contributor for the Ratio Contour project, a salient object detection and segmentation method used for computer vision applications.  Developed in \skill{C} and \skill{MATLAB}.}{github.com/malloc47/ratio-contour}

%     % \proj{Digital Collation}{Research project to ``collate'' high-resolution documents by using image registration, accomplished using the SIFT feature detector and a thin plate spline warping technique, written in MATLAB.}{github.com/malloc47/digital-collation}

%     % \proj{PMLDAP}{\skill{Linux} user management tool for Linux clusters.  Created as a simplified replacement for LDAP.  Capable of bootstrapping new systems, synchronizing users and configuration files, and running distributed commands.  Written in \skill{Bash}.}{github.com/malloc47/pmldap}

%     % \proj{matscicut}{An energy minimization framework for segmenting 3D materials volumes. Prototype of dissertation work, created in C++ using OpenCV libraries, with assorted MATLAB helper utilities.}{github.com/malloc47/matscicut}

%     % \proj{git-hq}{A remote management system for git, created in Python.}{github.com/malloc47/git-hq}

%     \projlh{Sina Weibo Mobile Client}{Created a \skill{J2ME}-based prototype mobile client for the popular Chinese \institution{Sina} microblogging service, similar to \institution{Twitter}.  Targeted at limited-functionality CLDC phones and uses a custom \skill{Java} wrapper for the \institution{Sina} API.  Employs symmetric-key encryption for personal data.}{http://bd.weibo.10086.cn/2012/downloads_kjava}{bd.weibo.10086.cn/2012/downloads\_kjav}

%   \end{longtable}























% \experience{2007--2008, 2011}%
% {Graduate Teaching Assistant at \textsc{USC}}%
% {Web Development}%
% {Supervised CSCE~145 labs, covering software development with \textsc{Java}, and taught CSCE~102, covering \textsc{Javascript}, \textsc{HTML}, and \textsc{CSS}. Taught~CSCE~211 covering digital logic design.}

% \experience{Spring 2007}%
% {Instructor for \textsc{CSCE 204} at \textsc{USCL}}%
% {Introductory Programming}%
% {Hired as special faculty. Taught introductory Visual Basic for majors and non-majors. Selected textbooks, developed all course material, graded all assignments. Worked with Dr. Noni M. Bohonak}

% \experience{Fall 2006}%
% {Camp Instructor for \textsc{USCL Arts and Sciences Adventure Camp}}%
% {5\textsuperscript{th}-8\textsuperscript{th} Grade Students}%
% {Worked in collaboration with Dr. Dwayne Brown. One of two instructors teaching Math and Computer Science to grade school students.}

% \experience{2003--2007}%
% {Professional Tutor at \textsc{USCL Academic Success Center}}%
% {High School and College Students}%
% {Student and graduate tutor for college-level Mathematics, Computer Science, Physics, and English classes.}
















% \section{Industry Experience}
% \vspace{-1em}
% \newcommand{\industry}[4]{
% \textsc{#1} & #2 &\emph{#3}\\
% &\multicolumn{2}{p{14cm}}{\footnotesize{#4}}\\
% \multicolumn{3}{c}{} \\ [-1ex]
% }

% \begin{longtable}{@{}r|p{10cm} r}

% \industry{2013\textemdash{}Present}%
% {Software Development Engineer III}%
% {\href{http://www.groupon.com}{Groupon, Inc.}}%%
% {Member of the Quantum Lead team building internal tools for
%     Groupon's sales workforce using \clojure to develop
%     service-oriented and big data systems. Built a specialized caching
%     system, rearchitected existing \hadoop pipeline in \cascalog, and
%     scaled a session management system used by a large
%     machine-learning data pipeline.}

% \industry{2012\textemdash{}2014}%
% {Technical Lead}%
% {\href{http://www.terrastride.com/}{TerraStride, Inc.}}%
% {Software developer in an agile startup environment creating the
%   \href{http://www.huntstand.com}{huntstand.com} web
%   application. Written using \python, \django, and \backbone; deployed
%   to \skill{AWS}.  Responsible for curating full technology stack and
%   coordinating with $5$ developers.}

% \industry{2011\textemdash{}2013}%
% {Project Manager}%
% {Palmetto Computer Labs}%
% {Assisted in planning the POSSCON conference. Managed the Open IT Lab
%   and associated projects (Android Development). Provided software
%   support for websites and managed projects.}

% \industry{2011}%
% {Contractor}%
% {Elastic Vision Consulting}%
% {Created a parser and generator for XML medical records formats (CCR
%   and CCD) in Java using JDOM, JAXB, SAX, Xerces, and Hibernate
%   (HSQLDB), on an Axis2+Jetty6 driven server.}

% \industry{2005}%
% {Intern --- Technical Writer}%
% {JAARS, Inc.}%
% {Created documentation and integrated context-sensitive online help
%   system for speech and linguistic software written in C++ and Visual
%   Basic.}

% \industry{2001\textemdash{}2002}%
% {Volunteer Software Developer}%
% {JAARS, Inc.}%
% {Spearheaded the conversion from VB4 to VB6 for the linguistic
%   reference tool
%   \href{http://www.sil.org/computing/ipahelp/ipaprvw2.htm}{IPA Help}.}

% \end{longtable}

% \pagebreak

% \section{Research Experience}
% \begin{longtable}{@{}r|p{14cm}}

% \experience{2011---2013}%
% {Research Assistant funded by \textsc{AFOSR}}%
% {Materials Volume Segmentation}%
% {Developed segmentation methods for materials image
%   volumes in \emph{Python+NumPy/SciPy} and \emph{MATLAB} at the
%   \textsc{Computer Vision Lab} at \textsc{USC}. Managed the lab
%   computer network and organized weekly lab meetings.  Created GUI
%   interface using wxWidgets for assisted segmentation, and conducted
%   large-scale evaluations on multiple datasets for metallic and
%   biological materials.}

% \experience{2010---2011}%
% {Research Assistant funded by \textsc{DARPA}}%
% {Video Event Recognition}%
% {Explored segmentation methods for video event
%   recognition. Attended P.I. meetings in San Diego (2010) and
%   Colorado (2011). Developed algorithms in \emph{Scheme} to process
%   a corpus of thousands of videos extracted into over 3 million
%   frames using a high-performance computing cluster.}

% \experience{2009---2010}%
% {NEH Fellow at the \textsc{Center for Digital Humanities}}%
% {Digital Collation}%
% {Created a \textsc{digital collation} application to
%   handle automatic differencing of sub-textual inconsistencies among
%   multiple copies of \emph{The Faerie Queene} by \textsc{Edmund
%     Spenser} in \emph{MATLAB} to process tens of thousands of book
%   page images.}

% \end{longtable}

% \pagestyle{myheadings}
% \markright{Jarrell Waggoner}