\newif\ifoutreach
% \outreachtrue % comment out to hide outreach

\newif\ifskills
%\skillstrue % comment out to hide skills section

\newif\ifreferees
%\refereestrue % comment out to hide referees



\documentclass[10pt]{article}

\usepackage{marvosym}
\usepackage{fontspec}
\usepackage{xunicode,xltxtra,url,parskip}
\defaultfontfeatures{Scale=MatchLowercase,Mapping=tex-text}
\RequirePackage{color,graphicx}
\usepackage[usenames,dvipsnames]{xcolor}
\usepackage[left=1in, right=1in, top=1in, bottom=1in]{geometry}
\usepackage{supertabular}
\usepackage{titlesec}
\usepackage{multicol}
\usepackage{multirow}
\usepackage{longtable}
\usepackage{xstring}
\usepackage{rotating}
\usepackage{ifthen}
\usepackage[xetex,
            unicode,
            pdfencoding=auto,
            pdfinfo={
              Title={João Faria / CV},
              Author={João Faria},
              Subject={João Faria CV},
              Keywords={astrophysics},
              Producer={xelatex},
              Creator{xelatex}
            },
            ]{hyperref}
\usepackage[absolute]{textpos}
\usepackage{enumitem}
\usepackage{tabularx}

% \usepackage[authoryear]{natbib}
% \usepackage[style=authoryear,natbib=true]{biblatex}
% \addbibresource{cv.bib}

% \makeatletter
% \renewcommand*{\@biblabel}[1]{\hfill[A#1]}
% \makeatother


\usepackage{etaremune}
\makeatletter
\long\def\thebibliography#1{%
  \section*{\refname}%
  \@mkboth{\MakeUppercase\refname}{\MakeUppercase\refname}
  \settowidth{\dimen0}{\@biblabel{#1}}%
  \setlength{\dimen2}{\dimen0}%
  \addtolength{\dimen2}{\labelsep}
  \sloppy
  \clubpenalty 4000 
  \@clubpenalty 
  \clubpenalty 
  \widowpenalty 4000
  \sfcode `\.\@m
  \renewcommand{\labelenumi}{\@biblabel{A\theenumi}} % labels like [A3], [A2], [A1]
  \begin{etaremune}[labelwidth=\dimen0,leftmargin=\dimen2]\@openbib@code
}
\def\endthebibliography{\end{etaremune}}
\def\@bibitem#1{%
  \item \if@filesw\immediate\write\@auxout{\string\bibcite{#1}{\the\value{enumi}}}\fi\ignorespaces
}
\makeatother


% Template obtained from http://www.cv-templates.info/2009/03/professional-cv-latex/

%Setup hyperref package, and colours for links
\definecolor{linkcolour}{rgb}{0,0.2,0.6}
\hypersetup{colorlinks,breaklinks,urlcolor=linkcolour, linkcolor=linkcolour}

%Color
\definecolor{lightg}{HTML}{999999}
\definecolor{medg}{HTML}{666666}
\definecolor{darkg}{HTML}{333333}

% Bullets
\definecolor{noteone}{HTML}{999999}
\definecolor{notetwo}{HTML}{848484}
\definecolor{notethree}{HTML}{424242}
\definecolor{notefour}{HTML}{212121}
\definecolor{notefive}{HTML}{000000}

\newcommand{\fivenotes}{%
	\textcolor{noteone}{\symbol{"2022}}
	\textcolor{notetwo}{\symbol{"2022}}
	\textcolor{notethree}{\symbol{"2022}}
	\textcolor{notefour}{\symbol{"2022}}
	\textcolor{notefive}{\symbol{"2022}}
}
\newcommand{\fournotes}{%
	\textcolor{noteone}{\symbol{"2022}}
	\textcolor{notetwo}{\symbol{"2022}}
	\textcolor{notethree}{\symbol{"2022}}
	\textcolor{notefour}{\symbol{"2022}}
	\textcolor{white}{\symbol{"2022}}
}
\newcommand{\threenotes}{%
	\textcolor{noteone}{\symbol{"2022}}
	\textcolor{notetwo}{\symbol{"2022}}
	\textcolor{notethree}{\symbol{"2022}}
	\textcolor{white}{\symbol{"2022}}
	\textcolor{white}{\symbol{"2022}}
}
\newcommand{\twonotes}{%
	\textcolor{noteone}{\symbol{"2022}}
	\textcolor{notetwo}{\symbol{"2022}}
	\textcolor{white}{\symbol{"2022}}
	\textcolor{white}{\symbol{"2022}}
	\textcolor{white}{\symbol{"2022}}
}
\newcommand{\onenote}{%
	\textcolor{noteone}{\symbol{"2022}}
	\textcolor{white}{\symbol{"2022}}
	\textcolor{white}{\symbol{"2022}}
	\textcolor{white}{\symbol{"2022}}
	\textcolor{white}{\symbol{"2022}}
}

\newcommand{\oneskill}{%
  \textcolor{white}{\symbol{"2022}}
  \textcolor{white}{\symbol{"2022}}
  \textcolor{notefive}{\symbol{"2022}}
}

\newcommand{\twoskill}{%
  \textcolor{white}{\symbol{"2022}}
  \textcolor{notethree}{\symbol{"2022}}
  \textcolor{notefive}{\symbol{"2022}}
}

\newcommand{\threeskill}{%
  \textcolor{noteone}{\symbol{"2022}}
  \textcolor{notethree}{\symbol{"2022}}
  \textcolor{notefive}{\symbol{"2022}}
}

%FONTS
% \defaultfontfeatures{Mapping=tex-text}
% \setmainfont[SmallCapsFont = Fontin SmallCaps]{Fontin}

\setromanfont [Ligatures={Common}, BoldFont={Linux Libertine Bold}, ItalicFont={Linux Libertine Italic}]{Linux Libertine}
\setsansfont [Ligatures={Common}, BoldFont={GeosansLight}, ItalicFont={GeosansLight}]{GeosansLight}
\setmonofont{GeosansLight} 

\font\lighttext=''Baskerville-Normal:color=787878'' at 10pt
\font\lighttextweb=''Baskerville-Normal:color=FF1493'' at 10pt

%CV Sections inspired by: 
%http://stefano.italians.nl/archives/26
\titleformat{\section}{\Large\scshape\raggedright\sffamily}{}{0em}{}[\titlerule]
%% \titlespacing{\section}{0pt}{-2pt}{0pt}

%-------------WATERMARK TEST [**not part of a CV**]---------------
\TPGrid[30mm,30mm]{30}{60}
%\setlength{\TPHorizModule}{30mm}
%\setlength{\TPVertModule}{\TPHorizModule}
%\textblockorigin{2mm}{0.65\paperheight}
\setlength{\parindent}{0pt}

\newcommand{\skill}{\textbf}
\newcommand{\institution}{\textsc}

% \def\bullet{\textcolor{medg}{\symbol{"00BB}}}
\def\div{\,\textbar{}\,}


\begin{document}
\pagestyle{empty}

% \font\fb=''[cmr10]''

\par{\centering {\Huge João P. \textsc{Faria} }\bigskip\par}

%% \section{Biographical Data}
%%
%% \begin{tabular}{r p{3.5in}}
%%   \textsc{Address:}	& Department of Computer Science and Engineering, University of South Carolina, Columbia, SC 29208 \\
%%   \textsc{Phone:}     & 847-261-4747\\
%%   \textsc{email:}     & \href{mailto:malloc47@gmail.com}{malloc47@gmail.com} \\
%%   \textsc{Website:}	& \href{http://www.malloc47.com}{www.malloc47.com} \\
%%   \textsc{Citizenship:} & United States Citizen \\
%% \end{tabular}


\begin{multicols}{2}
\setlength{\parskip}{0pt}
  \section{Contact}

\begin{tabularx}{\linewidth}{@{}l X@{}}
  \textsc{Address} & \small{***institute**} \\
                   & \small{***address**} \\
  \textsc{Phone}   & \small{***phone**}\\
  \textsc{email}   & \href{mailto:***email**}{***email**} \\
\end{tabularx}

\vfill
\columnbreak

\section{Online}
\begin{tabularx}{\linewidth}{@{}l X@{}}
  \textsc{Website}  & \href{***websiteescaped**}{***website**} \\
  \textsc{github}   & \href{http://www.github.com/***github**}{github.com/***github**}\\
  \textsc{Twitter}  & \href{https://twitter.com/***twitter**}{@***twitter**} \\
  \textsc{Orcid}    & \href{http://orcid.org/***orcid**}{***orcid**} \\
  \\
  % \textsc{LinkedIn}    & \href{http://www.linkedin.com/in/}{linkedin.com/} \\
\end{tabularx}

\end{multicols}

\section{Research Interests}

***interests**


\newcommand{\degree}[6]{\textsc{#1} & #2 in \textsc{#3} & \textbf{#4}\\
                        & \ifthenelse{\equal{#5}{}}{\small ~}{\small Advisors: #5} & \\
                        & \ifthenelse{\equal{#6}{}}{~}{\footnotesize Dissertation: ``#6''} & \\[1ex]
                        }

\section{Education}
\begin{tabularx}{\linewidth}{@{}r X l}

  % &\normalsize \textsc{GPA}: 3.8/4.0 | \small\emph{magna cum laude} \\[1ex]%

  ***degrees**

\end{tabularx}

\vspace{-2em}

\newcommand{\experience}[4]{
\textsc{#1} & #2 \\
\nopagebreak &\emph{#3}\\
\nopagebreak &\footnotesize{#4} \\
\nopagebreak \multicolumn{2}{c}{} \\ [-1ex]
}

% \section{Industry Experience}
% \vspace{-1em}
% \newcommand{\industry}[4]{
% \textsc{#1} & #2 &\emph{#3}\\
% &\multicolumn{2}{p{14cm}}{\footnotesize{#4}}\\
% \multicolumn{3}{c}{} \\ [-1ex]
% }

% \begin{longtable}{@{}r|p{10cm} r}

% \industry{2013\textemdash{}Present}%
% {Software Development Engineer III}%
% {\href{http://www.groupon.com}{Groupon, Inc.}}%%
% {Member of the Quantum Lead team building internal tools for
%     Groupon's sales workforce using \clojure to develop
%     service-oriented and big data systems. Built a specialized caching
%     system, rearchitected existing \hadoop pipeline in \cascalog, and
%     scaled a session management system used by a large
%     machine-learning data pipeline.}

% \industry{2012\textemdash{}2014}%
% {Technical Lead}%
% {\href{http://www.terrastride.com/}{TerraStride, Inc.}}%
% {Software developer in an agile startup environment creating the
%   \href{http://www.huntstand.com}{huntstand.com} web
%   application. Written using \python, \django, and \backbone; deployed
%   to \skill{AWS}.  Responsible for curating full technology stack and
%   coordinating with $5$ developers.}

% \industry{2011\textemdash{}2013}%
% {Project Manager}%
% {Palmetto Computer Labs}%
% {Assisted in planning the POSSCON conference. Managed the Open IT Lab
%   and associated projects (Android Development). Provided software
%   support for websites and managed projects.}

% \industry{2011}%
% {Contractor}%
% {Elastic Vision Consulting}%
% {Created a parser and generator for XML medical records formats (CCR
%   and CCD) in Java using JDOM, JAXB, SAX, Xerces, and Hibernate
%   (HSQLDB), on an Axis2+Jetty6 driven server.}

% \industry{2005}%
% {Intern --- Technical Writer}%
% {JAARS, Inc.}%
% {Created documentation and integrated context-sensitive online help
%   system for speech and linguistic software written in C++ and Visual
%   Basic.}

% \industry{2001\textemdash{}2002}%
% {Volunteer Software Developer}%
% {JAARS, Inc.}%
% {Spearheaded the conversion from VB4 to VB6 for the linguistic
%   reference tool
%   \href{http://www.sil.org/computing/ipahelp/ipaprvw2.htm}{IPA Help}.}

% \end{longtable}

% \pagebreak

% \section{Research Experience}
% \begin{longtable}{@{}r|p{14cm}}

% \experience{2011---2013}%
% {Research Assistant funded by \textsc{AFOSR}}%
% {Materials Volume Segmentation}%
% {Developed segmentation methods for materials image
%   volumes in \emph{Python+NumPy/SciPy} and \emph{MATLAB} at the
%   \textsc{Computer Vision Lab} at \textsc{USC}. Managed the lab
%   computer network and organized weekly lab meetings.  Created GUI
%   interface using wxWidgets for assisted segmentation, and conducted
%   large-scale evaluations on multiple datasets for metallic and
%   biological materials.}

% \experience{2010---2011}%
% {Research Assistant funded by \textsc{DARPA}}%
% {Video Event Recognition}%
% {Explored segmentation methods for video event
%   recognition. Attended P.I. meetings in San Diego (2010) and
%   Colorado (2011). Developed algorithms in \emph{Scheme} to process
%   a corpus of thousands of videos extracted into over 3 million
%   frames using a high-performance computing cluster.}

% \experience{2009---2010}%
% {NEH Fellow at the \textsc{Center for Digital Humanities}}%
% {Digital Collation}%
% {Created a \textsc{digital collation} application to
%   handle automatic differencing of sub-textual inconsistencies among
%   multiple copies of \emph{The Faerie Queene} by \textsc{Edmund
%     Spenser} in \emph{MATLAB} to process tens of thousands of book
%   page images.}

% \end{longtable}

% \pagestyle{myheadings}
% \markright{Jarrell Waggoner}


\let\originalbibitem\bibitem
\def\bibitem#1#2\par{%
  \noexpandarg
  \originalbibitem{#1}
  \StrSubstitute{#2}{J.~P. {Faria}}{\textbf{J.~P. Faria}}\par}

\nocite{*}

\renewcommand\refname{Publications}
\bibliography{cv}
% \bibliographystyle{aa}
\bibliographystyle{plainyr-rev}

% \printbibliography

\section{Talks / Posters}
\begin{enumerate}
\renewcommand{\labelenumi}{[P\arabic{enumi}] }

  ***postersANDtalks**

\end{enumerate}


\section{Participation in Conferences}
\begin{itemize}

  ***conferences**

\end{itemize}



\section{Teaching Experience}
\vspace{-1em}
\begin{longtable}{@{}r|p{14cm}}

% \experience{July 17-27, 2016}%
% {Detecting Planetary Transits and Radial-Velocity Signals}
% {IVth Azores International Advanced School in Space Sciences}
% {4.5hrs tutorial in the Summer school ``Asteroseismology and Exoplanets: Listening to the Stars\newline and Searching for New Worlds'' covering data analysis methods for the detection of exoplanets \newline in transit lightcurves and radial-velocity datasets.}

<<<<<<< HEAD
\experience{February, 2016}%
{Bayesian Statistics}%
{Advanced Course, Institute of Astrophysics and Space Sciences}%
{Created and lectured a course about Bayesian statistics, \newline composed of a seminar talk and two hands-on computational classes.}


\experience{November, 2015}%
{Python for Astronomers}%
{Advanced Course, Center for Astrophysics University of Porto}%
{Teaching assistant for introductory practical classes.}
=======

\experience{February, 2016}%
{Bayesian Statistics}%
{Advanced Course, Institute of Astrophysics and Space Sciences}%
{A course about Bayesian statistics, composed of a seminar talk and two hands-on computational classes (5hrs).}
>>>>>>> c3daa48a549c41fbdb30af35488a1d4233b1fe78


\experience{Aug 31 - Sep 4, 2015}%
{Escola de Verão de Física (Physics Summer School)}%
{Summer School, Faculty of Sciences, University of Porto}%
{Supervised and lectured a project about ``The energy of stars'' for a class of 5 high-school students.}

\experience{March 26-28, 2014}%
{Stellar evolution models with the MESA code}%
{Advanced Course, Center for Astrophysics University of Porto}%
{Developed and coordinated a series of three hands-on tutorial classes 
demonstrating the use\newline of the~\texttt{MESA} stellar evolution code 
(Paxton et al. \href{http://dx.doi.org/10.1088/0067-0049/192/1/3}{2011};
               \href{http://dx.doi.org/10.1088/0067-0049/208/1/4}{2013}).}

% \experience{2007--2008, 2011}%
% {Graduate Teaching Assistant at \textsc{USC}}%
% {Web Development}%
% {Supervised CSCE~145 labs, covering software development with \textsc{Java}, and taught CSCE~102, covering \textsc{Javascript}, \textsc{HTML}, and \textsc{CSS}. Taught~CSCE~211 covering digital logic design.}

% \experience{Spring 2007}%
% {Instructor for \textsc{CSCE 204} at \textsc{USCL}}%
% {Introductory Programming}%
% {Hired as special faculty. Taught introductory Visual Basic for majors and non-majors. Selected textbooks, developed all course material, graded all assignments. Worked with Dr. Noni M. Bohonak}

% \experience{Fall 2006}%
% {Camp Instructor for \textsc{USCL Arts and Sciences Adventure Camp}}%
% {5\textsuperscript{th}-8\textsuperscript{th} Grade Students}%
% {Worked in collaboration with Dr. Dwayne Brown. One of two instructors teaching Math and Computer Science to grade school students.}

% \experience{2003--2007}%
% {Professional Tutor at \textsc{USCL Academic Success Center}}%
% {High School and College Students}%
% {Student and graduate tutor for college-level Mathematics, Computer Science, Physics, and English classes.}

\end{longtable}

\ifoutreach

\section{Outreach}
\begin{itemize}
% \renewcommand{\labelenumi}{[G\arabic{enumi}] }
%% \item \emph{Combining Global Labeling and Local Relabeling for Metallic Image Segmentation}. Graduate Student Day Competition, Second Place. April 8, 2011.
\item \emph{Mentes que Brilham (TV show)}, interview about the work published in \href{http://adsabs.harvard.edu/cgi-bin/nph-abs_connect?fforward=http://dx.doi.org/10.1051/0004-6361/201425298}{Martins et al. (2015)}, May 13, 2015.
\item \emph{Universidade Júnior (University of Porto)}, monitor of scientific activities for middle-school students, July 2013.
\item \emph{Planetarium sessions (presenter)}, as collaborator of the Outreach Unit of CAUP, Summer 2013.
\item \emph{Noites no Observatório}, montlhy outreach events at the Lisbon Astronomical Observatory, 2010-2011.
% \item \emph{Aspect-Oriented Programming}. In CSCE 531. Guest lecture for Dr. Marco Valtorta. March 19, 2008.
\end{itemize}

\fi


%\section{Travel}
%\begin{enumerate}
%\renewcommand{\labelenumi}{[T\arabic{enumi}] }
%\item \emph{Mind's Eye PI Meeting}. DARPA Project P.I.~meeting. Denver, CO. January 20---21, 2011.
%\item \emph{Tomography and its Applications to Materials Science and Non-Destructive Evaluation}. Organizd by M. De Graef, L. Drummy, J. Simmons, M. Comer, C. Bouman, and J. Knopp. Tech\^{}Edge, Dayton, Ohio. December 13---15, 2010.
%\item Visiting scholar. In collaboration with J. M. Siskind. Purdue University, West Lafayette, IN. December 6---22, 2010 \& January 5---16, 2011.
%\item \emph{Mind's Eye Kickoff Meeting}. DARPA Project P.I.~meeting. San Diego, CA. September 23---24, 2010.
%\end{enumerate}

% \section{Honors/Awards}
% \begin{center}
% \begin{tabular*}{0.75\textwidth}{rll}
% 2012 & Gamecock Computing Research Symposium Poster Session,  First Place & \\
%      & Graduate Student Day Presentation,  First Place \\
% 2011 & Graduate Student Day Presentation,  Second Place \\
% 2010 & Graduate Student Day Presentation,  Honorable Mention \\
% 2009 & Upsilon Pi Epsilon \\
% % \multicolumn{3}{r}{}\\
% % 2006 & Senior Computer Science Award & {\lighttext \textcolor{lightg}{Bryan College}}\\
% \multicolumn{3}{r}{}\\
% 2004 & Clara P. Hammond Award & \\
% & Science and Mathematics Award \\
% & Highest Academic Average Award \\
% \end{tabular*}
% \end{center}

%\section{Professional Societies}

% \section{Classes Taught}
% \vspace{-1em}
% \begin{center}
% %\begin{tabular*}{0.75\textwidth}{r @{\hspace{0.5em}\textcolor{lightg}{\symbol{"00BB}}\hspace{0.5em}} l @{\extracolsep{\fill}} l }
% \begin{tabular*}{0.75\textwidth}{r @{\hspace{0.5em}\textcolor{lightg}{\symbol{"00BB}}\hspace{0.5em}} l l c }
% %\multicolumn{3}{r}{University of South Carolina}\\
% 2012--2013 & Open Source 101 & Open Source Software & \\
% 2012--2013 & Version Control 101 & git, github \\
% 2012--2013 & Command Line 101 & Linux, BASH \\
% \multicolumn{3}{r}{}\\
% Fall 2011 & CSCE 211 & Digital Logic Design &  \\
% Summer II 2008 & CSCE 102 & HTML/CSS/JavaScript \\
% Spring 2008 & CSCE 145 Lab & Java \\
% Fall 2007 & CSCE 145 Lab & Java \\
% \multicolumn{3}{r}{}\\
% %\multicolumn{3}{r}{University of South Carolina at Lancaster}\\ \hline
% Spring 2007 & CSCE 204 & Visual Basic & \\
% Spring 2007 & Math 241 \& Math 242 & Maple \\
% \end{tabular*}
% \end{center}

% \vspace{2em}

% \section{Service}
% \vspace{-1em}
% \newcommand{\service}[2]{
%   \textsc{#1} & #2\\
%   %% \multicolumn{2}{c}{} \\
% }
% \begin{longtable}{r|p{10cm}}
%   \service{Itern mentoring}{Groupon internship program, 2014}
%   \service{Webmaster}{\href{http://cvl.cse.sc.edu/wvm2013/}{Winter Vision Meetings, 2013}}
%   \service{Webmaster}{\href{http://cvl.cse.sc.edu/wacv2013/}{Workshop on the Applications of Computer Vision, 2013}}
%   \service{Judge}{Discovery Day --- Undergraduate Research Presentations}
%   \service{Reviewer}{Pattern Recognition Letters}
%   \service{Reviewer}{IEEE Transactions on Pattern Analysis and Machine Intelligence}
%   %% \service{Fellow}{NSF GK-12 Program}
%   \service{Member}{Institute of Electrical and Electronics Engineers (IEEE)}
%   \service{SysAdmin}{Computer Vision Lab}
% \end{longtable}

% \vspace{1em}

% \section{Personal and Open Source Projects}
%   \newcommand{\proj}[3]{
%     \textsc{#1} & #2\\
%      &\href{http://www.#3}{#3}\\
%      \multicolumn{2}{c}{} \\ [-1.5ex]
%   }

%   \newcommand{\projlh}[4]{
%     \textsc{#1} & #2\\
%      &\href{#3}{#4}\\
%   }

%   \begin{longtable}{@{}p{3cm}|p{13cm}}

%     \proj{matsciseg}%
%     {Framework for propagated 3D volume segmentation, used in my dissertation work.  Algorithms created in \python and \cpp and exposed as a web API using \django. Includes a web application that consumes the API created in \js, and \jquery.}%
%     {github.com/malloc47/matsciseg}

%   %   \proj{\href{http://nonpartisan.me}{nonpartisan.me}}%
%   % {Google Chrome extension that filters social media websites for political keywords.  Available in the \href{https://chrome.google.com/webstore/detail/nonpartisanme/ninebcppidndhampaggnjbijpacoadgg}{Chrome Web Store}.  Featured in the \href{http://www.charlestoncitypaper.com/charleston/sick-of-politics-on-facebook-try-this-browser-tool/Content?oid=4153447}{Charleston City Paper}.}%
%   % {github.com/malloc47/nonpartisan.me}

%   % \proj{term-do}{An interactive terminal prompt that displays potential command completions as you type.  A hybrid of gnome-do and Emacs's ido-mode.  Works on many tested VT100 terminal types; built in~\skill{C++}.  Includes client/server architecture implemented with boost.interprocess and full-featured plugin system. Available in the \href{https://aur.archlinux.org/packages/term-do-git/}{Arch Linux AUR}.}{github.com/malloc47/term-do}

%     % \proj{Ratio Contour}{Maintainer and contributor for the Ratio Contour project, a salient object detection and segmentation method used for computer vision applications.  Developed in \skill{C} and \skill{MATLAB}.}{github.com/malloc47/ratio-contour}

%     % \proj{Digital Collation}{Research project to ``collate'' high-resolution documents by using image registration, accomplished using the SIFT feature detector and a thin plate spline warping technique, written in MATLAB.}{github.com/malloc47/digital-collation}

%     % \proj{PMLDAP}{\skill{Linux} user management tool for Linux clusters.  Created as a simplified replacement for LDAP.  Capable of bootstrapping new systems, synchronizing users and configuration files, and running distributed commands.  Written in \skill{Bash}.}{github.com/malloc47/pmldap}

%     % \proj{matscicut}{An energy minimization framework for segmenting 3D materials volumes. Prototype of dissertation work, created in C++ using OpenCV libraries, with assorted MATLAB helper utilities.}{github.com/malloc47/matscicut}

%     % \proj{git-hq}{A remote management system for git, created in Python.}{github.com/malloc47/git-hq}

%     \projlh{Sina Weibo Mobile Client}{Created a \skill{J2ME}-based prototype mobile client for the popular Chinese \institution{Sina} microblogging service, similar to \institution{Twitter}.  Targeted at limited-functionality CLDC phones and uses a custom \skill{Java} wrapper for the \institution{Sina} API.  Employs symmetric-key encryption for personal data.}{http://bd.weibo.10086.cn/2012/downloads_kjava}{bd.weibo.10086.cn/2012/downloads\_kjav}

%   \end{longtable}

\vspace{2em}

\ifskills

\begin{minipage}{\linewidth}
  \newcommand{\skills}[2]{
    \item #2 #1
  }
  \section{Skills \& Languages}
    \begin{multicols}{4}
      \raggedcolumns
      \begin{itemize}
        \renewcommand{\labelitemi}{}
        \renewcommand{\skill}{\textnormal}
        \setlength{\itemsep}{1pt}
        \setlength{\parskip}{0pt}
        \setlength{\parsep}{0pt}

        % \skills{\bash}{\oneskill}
        \skills{\python}{\threeskill}
        \skills{\ccpp}{\twoskill}
        %% \skills{English}{\threeskill}
        \skills{\git}{\threeskill}
        \skills{Fortran}{\threeskill}
        \skills{GNU/\linux}{\threeskill}
        %% \skills{\html}{\threeskill}
        \skills{\LaTeX}{\twoskill}
        %\skills{Learning}{\fournotes Machine}
        \skills{\matlab}{\oneskill}

      \end{itemize}
    \end{multicols}
    \begin{footnotesize}
      \raggedleft
        Large-scale and/or multi-group projects ~ \threeskill \\
        Multiple projects and/or experience teaching \twoskill \\
        Small projects or assignments \oneskill \\
    \end{footnotesize}

    \begin{center}
    \noindent\rule{8cm}{0.4pt}
    \end{center}

    \begin{multicols}{4}
      \begin{itemize}
      \renewcommand{\labelitemi}{}
      \setlength{\itemsep}{1pt}
      \setlength{\parskip}{0pt}
      \setlength{\parsep}{0pt}

      \skills{Portuguese}{\threeskill}
      \skills{French}{\oneskill}
      \skills{English}{\twoskill}
      \skills{Spanish}{\oneskill}
      \skills{~}{}
      \skills{~}{}
      \end{itemize}
    \end{multicols}
    \begin{minipage}{0.5\linewidth}
      \begin{footnotesize}
      \raggedleft
         Native speaker \threeskill \\
        Profficient \twoskill \\
        Basic understanding \oneskill \\
      \end{footnotesize}
    \end{minipage}

\end{minipage}

\fi

% \section{Activities}
% teaching, programming, open source software, system administration,
% data visualization, Linux,
% \href{https://soundcloud.com/malloc47}{music composition}

\ifreferees

\vspace{3in}
{\Large\scshape\raggedright\sffamily Referees}
\newcommand{\service}[2]{
  \textsc{#1} & #2\\[1em]
  %% \multicolumn{2}{c}{} \\
}
\begin{longtable}{r|p{10cm}}
  \service{Nuno C. Santos}{\ldots}
  \service{Pedro Figueira}{\ldots}
\end{longtable}

\fi

\null\vfill
\footnotesize{
  Last updated: \today \quad---\quad
  Online:  \href{http://j-faria.github.io/cv}{\textasciitilde jfaria/cv} \quad---\quad
  % Résumé: \href{http://resume.malloc47.com}{resume.malloc47.com} \hfill
  Source:  \href{https://github.com/j-faria/cv}{github.com/j-faria/cv/}
}

\end{document}
