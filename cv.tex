\documentclass[10pt]{article}

\usepackage{marvosym}
\usepackage{fontspec}
\usepackage{xunicode,xltxtra,url,parskip}
\defaultfontfeatures{Scale=MatchLowercase,Mapping=tex-text}
\RequirePackage{color,graphicx}
\usepackage[usenames,dvipsnames]{xcolor}
\usepackage[left=1in, right=1in, top=1in, bottom=1in]{geometry}
\usepackage{supertabular}
\usepackage{titlesec}
\usepackage{multicol}
\usepackage{multirow}
\usepackage{longtable}
\usepackage{xstring}
\usepackage{rotating}
\usepackage{ifthen}
\usepackage[xetex,
            pdfauthor={Jarrell Waggoner},
            pdftitle={malloc47/cv},
            pdfsubject={Jarrell Waggoner CV},
            pdfkeywords={computer vision, image processing, artificial intelligence, pattern recognition, machine learning, data science, functional programming, web development},
            pdfproducer={xelatex},
            pdfcreator={xelatex}]{hyperref}
\usepackage[absolute]{textpos}
\usepackage{enumitem}
\usepackage{tabularx}

\makeatletter
\renewcommand*{\@biblabel}[1]{\hfill[C#1]}
\makeatother

% Template obtained from http://www.cv-templates.info/2009/03/professional-cv-latex/

%Setup hyperref package, and colours for links
\definecolor{linkcolour}{rgb}{0,0.2,0.6}
\hypersetup{colorlinks,breaklinks,urlcolor=linkcolour, linkcolor=linkcolour}

%Color
\definecolor{lightg}{HTML}{999999}
\definecolor{medg}{HTML}{666666}
\definecolor{darkg}{HTML}{333333}

% Bullets
\definecolor{noteone}{HTML}{999999}
\definecolor{notetwo}{HTML}{848484}
\definecolor{notethree}{HTML}{424242}
\definecolor{notefour}{HTML}{212121}
\definecolor{notefive}{HTML}{000000}

\newcommand{\fivenotes}{%
	\textcolor{noteone}{\symbol{"2022}}
	\textcolor{notetwo}{\symbol{"2022}}
	\textcolor{notethree}{\symbol{"2022}}
	\textcolor{notefour}{\symbol{"2022}}
	\textcolor{notefive}{\symbol{"2022}}
}
\newcommand{\fournotes}{%
	\textcolor{noteone}{\symbol{"2022}}
	\textcolor{notetwo}{\symbol{"2022}}
	\textcolor{notethree}{\symbol{"2022}}
	\textcolor{notefour}{\symbol{"2022}}
	\textcolor{white}{\symbol{"2022}}
}
\newcommand{\threenotes}{%
	\textcolor{noteone}{\symbol{"2022}}
	\textcolor{notetwo}{\symbol{"2022}}
	\textcolor{notethree}{\symbol{"2022}}
	\textcolor{white}{\symbol{"2022}}
	\textcolor{white}{\symbol{"2022}}
}
\newcommand{\twonotes}{%
	\textcolor{noteone}{\symbol{"2022}}
	\textcolor{notetwo}{\symbol{"2022}}
	\textcolor{white}{\symbol{"2022}}
	\textcolor{white}{\symbol{"2022}}
	\textcolor{white}{\symbol{"2022}}
}
\newcommand{\onenote}{%
	\textcolor{noteone}{\symbol{"2022}}
	\textcolor{white}{\symbol{"2022}}
	\textcolor{white}{\symbol{"2022}}
	\textcolor{white}{\symbol{"2022}}
	\textcolor{white}{\symbol{"2022}}
}

\newcommand{\oneskill}{%
  \textcolor{white}{\symbol{"2022}}
  \textcolor{white}{\symbol{"2022}}
  \textcolor{notefive}{\symbol{"2022}}
}

\newcommand{\twoskill}{%
  \textcolor{white}{\symbol{"2022}}
  \textcolor{notethree}{\symbol{"2022}}
  \textcolor{notefive}{\symbol{"2022}}
}

\newcommand{\threeskill}{%
  \textcolor{noteone}{\symbol{"2022}}
  \textcolor{notethree}{\symbol{"2022}}
  \textcolor{notefive}{\symbol{"2022}}
}

%FONTS
% \defaultfontfeatures{Mapping=tex-text}
% \setmainfont[SmallCapsFont = Fontin SmallCaps]{Fontin}

\setromanfont [Ligatures={Common}, BoldFont={Linux Libertine Bold}, ItalicFont={Linux Libertine Italic}]{Linux Libertine}
\setsansfont [Ligatures={Common}, BoldFont={GeosansLight}, ItalicFont={GeosansLight}]{GeosansLight}
\setmonofont{GeosansLight} 

\font\lighttext=''Baskerville-Normal:color=787878'' at 10pt
\font\lighttextweb=''Baskerville-Normal:color=FF1493'' at 10pt

%CV Sections inspired by: 
%http://stefano.italians.nl/archives/26
\titleformat{\section}{\Large\scshape\raggedright\sffamily}{}{0em}{}[\titlerule]
%% \titlespacing{\section}{0pt}{-2pt}{0pt}

%-------------WATERMARK TEST [**not part of a CV**]---------------
\TPGrid[30mm,30mm]{30}{60}
%\setlength{\TPHorizModule}{30mm}
%\setlength{\TPVertModule}{\TPHorizModule}
%\textblockorigin{2mm}{0.65\paperheight}
\setlength{\parindent}{0pt}

\newcommand{\skill}{\textbf}
\newcommand{\institution}{\textsc}

% \def\bullet{\textcolor{medg}{\symbol{"00BB}}}
\def\div{\,\textbar{}\,}


\begin{document}
\pagestyle{empty}

% \font\fb=''[cmr10]''

\par{\centering {\Huge Jarrell \textsc{Waggoner} }\bigskip\par}

%% \section{Biographical Data}
%%
%% \begin{tabular}{r p{3.5in}}
%%   \textsc{Address:}	& Department of Computer Science and Engineering, University of South Carolina, Columbia, SC 29208 \\
%%   \textsc{Phone:}     & 847-261-4747\\
%%   \textsc{email:}     & \href{mailto:malloc47@gmail.com}{malloc47@gmail.com} \\
%%   \textsc{Website:}	& \href{http://www.malloc47.com}{www.malloc47.com} \\
%%   \textsc{Citizenship:} & United States Citizen \\
%% \end{tabular}


\begin{multicols}{2}
\setlength{\parskip}{0pt}
  \section{Biographical}

\begin{tabularx}{\linewidth}{@{}l X@{}}
  \textsc{Address}	& \footnotesize{Department of Computer Science and Engineering,} \\
  & \footnotesize{University of South Carolina, Columbia, SC 29208} \\
  \textsc{Phone}       & 847-261-4747\\
  \textsc{email}       & \href{mailto:jarrell.waggoner@gmail.com}{jarrell.waggoner@gmail.com} \\
\end{tabularx}

\vfill
\columnbreak

\section{Online}
\begin{tabularx}{\linewidth}{@{}l X@{}}
  \textsc{Website}	& \href{http://www.malloc47.com}{www.malloc47.com} \\
  \textsc{Twitter}     & \href{https://twitter.com/malloc47}{@malloc47} \\
  \textsc{github}      & \href{http://www.github.com/malloc47}{github.com/malloc47}\\
  \textsc{LinkedIn}    & \href{http://www.linkedin.com/in/malloc47}{linkedin.com/in/malloc47} \\
\end{tabularx}

\end{multicols}

\vspace{2em}

\section{Research Interests}

computer vision, segmentation, contour completion, perceptual
grouping, document image analysis, event recognition, image
processing, artificial intelligence, pattern recognition \& machine
learning, data science, functional programming

\vspace{2em}

\section{Education}
\begin{tabular}{r p{12cm}}
  \textsc{Present} & Ph.D. Candidate in \textsc{Computer Science}, \textbf{University of South Carolina}\\
  &% Dissertation: ``Combining Global Labeling and Local Relabeling for Metallic Image Segmentation'' |
  \small Advisor: Dr. Song \textsc{Wang} % | \normalsize \textsc{Gpa}: 3.85/4.0
%\hyperlink{grds}{\hfill | \footnotesize Detailed List of Exams}
  \\% &\\
  \textsc{May} 2009 & Master of Engineering in \textsc{Computer Science}, \textbf{University of South Carolina}\\
  &\normalsize \textsc{GPA}: 3.8/4.0 | \small\emph{magna cum laude}
%\hyperlink{grds}{\hfill | \footnotesize Detailed List of Exams}
\\% &\\
\textsc{May} 2006& Bachelor of Science in \textsc{Computer Science}, \textbf{Bryan College} \\
%& Thesis: ``Aspect-Oriented Programming in an Extreme Programming Environment''\\
&\normalsize 
%\textsc{GPA}: ?.?/4.0 | 
\small\emph{summa cum laude}
%\hyperlink{grds_cleli}{\hfill| \footnotesize Detailed List of Exams}
\\% &\\
\textsc{May} 2004& Associate of Science in \textsc{Computer Science}\\
&\textbf{University of South Carolina at Lancaster} \\
&\normalsize \textsc{GPA}: 4.0/4.0 | \small\emph{summa cum laude}\\
%\hyperlink{grds_cleli}{\hfill| \footnotesize Detailed List of Exams}
\end{tabular}

\vspace{2em}

\newcommand{\experience}[4]{
\textsc{#1} & #2 \\
\nopagebreak &\emph{#3}\\
\nopagebreak &\footnotesize{#4} \\
\nopagebreak \multicolumn{2}{c}{} \\
}

\section{Research Experience}
\begin{longtable}{r|p{11cm}}

\experience{2011---Present}%
{Research Assistant funded by \textsc{AFOSR}}%
{Materials Volume Segmentation}%
{Developed segmentation methods for materials image
  volumes in \emph{Python+NumPy/SciPy} and \emph{MATLAB} at the
  \textsc{Computer Vision Lab} at \textsc{USC}. Managed the lab
  computer network and organized weekly lab meetings.  Created GUI
  interface using wxWidgets for assisted segmentation, and conducted
  large-scale evaluations on multiple datasets for metallic and
  biological materials.}
%% {Developed segmentation methods for materials image volumes.  Created GUI interface for assisted segmentation, and conducted large-scale evaluations on multiple datasets for metallic and biological materials.}

\experience{2010---2011}%
{Research Assistant funded by \textsc{DARPA}}%
{Video Event Recognition}%
{Explored segmentation methods for video event
  recognition. Attended P.I. meetings in San Diego (2010) and
  Colorado (2011). Developed algorithms in \emph{Scheme} to process
  a corpus of thousands of videos extracted into over 3 million
  frames using a high-performance computing cluster.}

%% {Explored segmentation methods for video event recognition while working at the \textsc{Computer Vision Lab} at \textsc{USC}. Managed lab computer network and organize weekly lab meetings.  Attended P.I. meetings in San Diego (2010) and Colorado (2011). Visited \textsc{Purdue University} working with Dr.~Jeffrey Mark Siskind (Dec~2010---Jan~2011).}

\experience{2009---2010}%
{NSF Fellow at the \textsc{Center for Digital Humanities}}%
{Digital Collation}%
{Created a \textsc{digital collation} application to
  handle automatic differencing of sub-textual inconsistencies among
  multiple copies of \emph{The Faerie Queene} by \textsc{Edmund
    Spenser} in \emph{MATLAB} to process tens of thousands of book
  page images.}
%% {Created a \textsc{digital collation} application to handle automatic differencing of sub-textual inconsistencies among multiple copies of \emph{The Faerie Queene} by \textsc{Edmund Spenser}.}

\end{longtable}

\section{Teaching Experience}
\begin{longtable}{r|p{11cm}}

\experience{2008--2009}%
{GK-12 Fellow at \textsc{Crayton Middle School}}%
{8\textsuperscript{th} Grade Science}%
{Served in Crayton Middle School, coordinating with the classroom instructor to enhance the science curriculum and activities in an 8\textsuperscript{th} grade science classroom. Subsequently coordinated and taught at the \textsc{GK-12 Institute for Teachers}, presenting the activities developed and delivered in the classroom.}

\experience{2007--2008, 2011}%
{Graduate Teaching Assistant at \textsc{USC}}%
{Web Development}%
{Supervised CSCE~145 labs, covering software development with \textsc{Java}, and taught CSCE~102, covering \textsc{Javascript}, \textsc{HTML}, and \textsc{CSS}. Taught~CSCE~211 covering digital logic design.}

\experience{Spring 2007}%
{Instructor for \textsc{CSCE 204} at \textsc{USCL}}%
{Introductory Programming}%
{Hired as special faculty. Taught introductory Visual Basic for majors and non-majors. Selected textbooks, developed all course material, graded all assignments. Worked with Dr. Noni M. Bohonak}

\experience{Fall 2006}%
{Camp Instructor for \textsc{USCL Arts and Sciences Adventure Camp}}%
{5\textsuperscript{th}-8\textsuperscript{th} Grade Students}%
{Worked in collaboration with Dr. Dwayne Brown. One of two instructors teaching Math and Computer Science to grade school students.}

\experience{2003--2007}%
{Professional Tutor at \textsc{USCL Academic Success Center}}%
{High School and College Students}%
{Student and graduate tutor for college-level Mathematics, Computer Science, Physics, and English classes.}

\end{longtable}

\pagestyle{myheadings}
\markright{Jarrell Waggoner}

\section{Industry Experience}

\newcommand{\industry}[4]{
\textsc{#1} & #2 &\emph{#3}\\
&\multicolumn{2}{p{11cm}}{\footnotesize{#4}}\\
\multicolumn{3}{c}{} \\
}

\begin{longtable}{r|p{7cm} r}

\industry{2012\textemdash{}Present}%
{Technical Lead}%
{Huntstand, Inc.}%
{Developed the \href{http://www.huntstand.com}{www.huntstand.com} web application using Python+Django with a PJAX frontend which was deployed to AWS;  responsible for curating full technology stack and coordinating with multiple developers.}

\industry{2011\textemdash{}Present}%
{Project Manager}%
{Palmetto Computer Labs}%
{Assisted in planning the POSSCON conference. Managed the Open IT Lab and associated projects (Android Development). Provided software support for websites and managed projects.}

\industry{2011}%
{Contractor}%
{Elastic Vision Consulting}%
{Created a parser and generator for XML medical records formats (CCR and CCD) in Java using JDOM, JAXB, SAX, Xerces, and Hibernate (HSQLDB), on an Axis2+Jetty6 driven server.}

\industry{2005}%
{Intern --- Technical Writer}%
{JAARS, Inc.}%
{Created documentation and integrated context-sensitive online help system for speech and linguistic software written in C++ and Visual Basic.}

\industry{2001\textemdash{}2002}%
{Volunteer Software Developer}%
{JAARS, Inc.}%
{Spearheaded the conversion from VB4 to VB6 for the linguistic reference tool \href{http://www.sil.org/computing/ipahelp/ipaprvw2.htm}{IPA Help}.}

\end{longtable}

%\section{Publications}
\nocite{waggoner:13a}
\nocite{waggoner:13b}
\nocite{waggoner:13c}
\nocite{waggoner:11}
\nocite{wang:11}
\nocite{temlyakov:10}
\nocite{zhang:10}
\nocite{waggoner:12}
\nocite{barbu:12}
\nocite{zhang:12}
\nocite{temlyakov:13}
\nocite{salvi:13a}
\nocite{salvi:13b}

\renewcommand\refname{Publications}
\bibliography{cv}
\bibliographystyle{plainyr-rev}

\section{Posters/Presentations}
\begin{enumerate}
\renewcommand{\labelenumi}{[P\arabic{enumi}] }
\item \emph{Homeomorphic Multi-Structure Propagation for Metallic
  Image Segmentation}.  Gamecock Computing Research Symposium.
  Columbia, SC.  October 5, 2012.
\item \emph{Android Application Development Workshop}.  Appathon
  Contest.  Columbia, SC.  November 17, 2012.
\item \emph{Open Source and Education}. SC Municipal Technology
  Association (SCMTA) Conference. Charleston, SC.  September 6, 2012.
\item \emph{Open Source and Higher Education}.  SC Technical College
  System (SCTCS) Conference.  Columbia, SC.  September 25, 2012.
\item \emph{Introduction to Android Development}.  Digital Humanities
  High Performance Computing (DHHPC) Workshop.  Columbia, SC.  August
  8, 2012.
\item \emph{Combining Global Labeling and Local Relabeling for
  Metallic Image Segmentation}.  SPIE (Computational Image X).
  Burlingame, CA.  January 23, 2012.
\item \emph{Open Source and Government}.  SC Government Management
  Information Systems (SCGMIS) Software Developers Workshop.
  Columbia, SC.  January 19, 2012.
\item \emph{Superpixel Contour Completion}.  DARPA Mind's Eye PI
  Meeting.  Denver, CO.  January 20, 2011.
\end{enumerate}

\vspace{2em}

\section{Guest Lectures}
\begin{enumerate}
\renewcommand{\labelenumi}{[G\arabic{enumi}] }
%% \item \emph{Combining Global Labeling and Local Relabeling for Metallic Image Segmentation}. Graduate Student Day Competition, Second Place. April 8, 2011.
%% \item \emph{Image Registration for Digital Collation}. Graduate Student Day Competition, Honorable Mention. April 2, 2010.
\item \emph{Building Chrome Extensions}.  In CSCE 242.  Guest lecture for Dr. José M. Vidal.  November~30, 2012.
\item \emph{Modeling in Blender}.  In CSCE 552.  Guest lecture for Dr. Jijun Tang.  February~28, 2011.
\item \emph{Aspect-Oriented Programming}. In CSCE 531. Guest lecture for Dr. Marco Valtorta. March 19, 2008.
\item \emph{Math 241}. Vector Calculus. Guest lecture for Dr. Dwayne Brown. April~23---26, 2007. 
\item \emph{Math 242}. Differential Equations. Guest lecture for Dr. Dwayne Brown. April~23---26, 2007. 
\end{enumerate}

%\section{Travel}
%\begin{enumerate}
%\renewcommand{\labelenumi}{[T\arabic{enumi}] }
%\item \emph{Mind's Eye PI Meeting}. DARPA Project P.I.~meeting. Denver, CO. January 20---21, 2011. 
%\item \emph{Tomography and its Applications to Materials Science and Non-Destructive Evaluation}. Organizd by M. De Graef, L. Drummy, J. Simmons, M. Comer, C. Bouman, and J. Knopp. Tech\^{}Edge, Dayton, Ohio. December 13---15, 2010.
%\item Visiting scholar. In collaboration with J. M. Siskind. Purdue University, West Lafayette, IN. December 6---22, 2010 \& January 5---16, 2011.
%\item \emph{Mind's Eye Kickoff Meeting}. DARPA Project P.I.~meeting. San Diego, CA. September 23---24, 2010. 
%\end{enumerate}

\vspace{2em}

\section{Honors/Awards}
\begin{tabular}{rll}
2012 & Gamecock Computing Research Symposium Poster Session,  First Place & \multirow{5}{*}{{\lighttext \textcolor{lightg}{\begin{turn}{-90}USC\end{turn}}}}\\
     & Graduate Student Day Presentation,  First Place \\
2011 & Graduate Student Day Presentation,  Second Place \\
2010 & Graduate Student Day Presentation,  Honorable Mention \\
2009 & Upsilon Pi Epsilon \\
\multicolumn{3}{r}{}\\
2006 & Senior Computer Science Award & {\lighttext \textcolor{lightg}{Bryan College}}\\
\multicolumn{3}{r}{}\\
2004 & Clara P. Hammond Award & \multirow{3}{*}{{\lighttext \textcolor{lightg}{\begin{turn}{-90}USCL\end{turn}}}} \\
& Science and Mathematics Award \\
& Highest Academic Average Award \\
\end{tabular}

%\section{Professional Societies}

\vspace{2em}

\section{Teaching}
\begin{center}
%\begin{tabular*}{0.75\textwidth}{r @{\hspace{0.5em}\textcolor{lightg}{\symbol{"00BB}}\hspace{0.5em}} l @{\extracolsep{\fill}} l }
\begin{tabular*}{0.75\textwidth}{r @{\hspace{0.5em}\textcolor{lightg}{\symbol{"00BB}}\hspace{0.5em}} l l c }
%\multicolumn{3}{r}{University of South Carolina}\\
Ongoing & Open Source 101 & Open Source Software & \multirow{3}{*}{{\lighttext \textcolor{lightg}{\begin{turn}{-90}IT-oLogy\end{turn}}}}  \\
Ongoing & Version Control 101 & git, github \\
Ongoing & Command Line 101 & Linux, BASH \\
\multicolumn{3}{r}{}\\
Fall 2011 & CSCE 211 & Digital Logic Design & \multirow{4}{*}{{\lighttext \textcolor{lightg}{\begin{turn}{-90}USC\end{turn}}}} \\
Summer II 2008 & CSCE 102 & HTML/CSS/Javasript \\
Spring 2008 & CSCE 145 Lab & Java \\
Fall 2007 & CSCE 145 Lab & Java \\
\multicolumn{3}{r}{}\\
%\multicolumn{3}{r}{University of South Carolina at Lancaster}\\ \hline
Spring 2007 & CSCE 204 & Visual Basic & \multirow{2}{*}{{\lighttext \textcolor{lightg}{\begin{turn}{-90}USCL\end{turn}}}} \\
Spring 2007 & Math 241 \& Math 242 & Maple \\
\end{tabular*}
\end{center}


\section{Service}
\newcommand{\service}[2]{
  \textsc{#1} & #2\\
  %% \multicolumn{2}{c}{} \\
}
\begin{longtable}{r|p{10cm}}
  \service{Webmaster}{\href{http://cvl.cse.sc.edu/wvm2013/}{Winter Vision Meetings, 2013}}
  \service{Webmaster}{\href{http://cvl.cse.sc.edu/wacv2013/}{Workshop on the Applications of Computer Vision, 2013}}
  \service{Judge}{Discovery Day --- Undergraduate Research Presentations}
  \service{Reviewer}{Pattern Recognition Letters}
  \service{Reviewer}{IEEE Transactions on Pattern Analysis and Machine Intelligence}
  %% \service{Fellow}{NSF GK-12 Program}
  \service{Member}{Institute of Electrical and Electronics Engineers (IEEE)}
  \service{SysAdmin}{Computer Vision Lab}
\end{longtable}

\section{Personal and Open Source Projects}
\newcommand{\proj}[3]{
  \textsc{#1} & #2\\
   &\href{http://www.#3}{#3}\\
   \multicolumn{2}{c}{} \\ [-1ex]
}
\begin{longtable}{r|p{13cm}}

  \proj{\href{http://nonpartisan.me}{nonpartisan.me}}%
{Google Chrome extension that filters social media websites for political keywords.  Available in the \href{https://chrome.google.com/webstore/detail/nonpartisanme/ninebcppidndhampaggnjbijpacoadgg}{Chrome Web Store}.  Featured in the \href{http://www.charlestoncitypaper.com/charleston/sick-of-politics-on-facebook-try-this-browser-tool/Content?oid=4153447}{Charleston City Paper}.}%
{github.com/malloc47/nonpartisan.me}

\proj{term-do}{An interactive terminal prompt that displays potential command completions as you type.  A hybrid of gnome-do and Emacs's ido-mode.  Works on many tested VT100 terminal types; built in~\skill{C++}.  Includes client/server architecture implemented with boost.interprocess and full-featured plugin system. Available in the \href{https://aur.archlinux.org/packages/term-do-git/}{Arch Linux AUR}.}{github.com/malloc47/term-do}

  \proj{Ratio Contour}{Maintainer and contributor for the Ratio Contour project, a salient object detection and segmentation method used for computer vision applications.  Developed in \skill{C} and \skill{MATLAB}.}{github.com/malloc47/ratio-contour}

  \proj{Digital Collation}{Research project to ``collate'' high-resolution documents by using image registration, accomplished using the SIFT feature detector and a thin plate spline warping technique, written in MATLAB.}{github.com/malloc47/digital-collation}

  \proj{PMLDAP}{\skill{Linux} user management tool for Linux clusters.  Created as a simplified replacement for LDAP.  Capable of bootstrapping new systems, synchronizing users and configuration files, and running distributed commands.  Written in \skill{Bash}.}{github.com/malloc47/pmldap}

  \proj{matscicut}{An energy minimization framework for segmenting 3D materials volumes. Prototype of dissertation work, created in C++ using OpenCV libraries, with assorted MATLAB helper utilities.}{github.com/malloc47/matscicut}

  \proj{git-hq}{A remote management system for git, created in Python.}{github.com/malloc47/git-hq}

\end{longtable}

\newcommand{\skills}[2]{
  \item #2 #1
}
\vspace{-0.5em}
\section{Skills \& Languages}
\vspace{-1em}
\begin{multicols}{4}
\raggedcolumns
\begin{itemize}
\renewcommand{\labelitemi}{}
\setlength{\itemsep}{1pt}
\setlength{\parskip}{0pt}
\setlength{\parsep}{0pt}

%% \skills{Assembly}{\twonotes}
\skills{Bash}{\threeskill}
\skills{Blender}{\twoskill}
\skills{C/C++}{\threeskill }
\skills{Emacs Lisp}{\oneskill}
\skills{English}{\threeskill}
\skills{git}{\twoskill}
\skills{GNU/Linux}{\threeskill}
\skills{Haskell}{\oneskill}
\skills{HTML/CSS}{\threeskill}
%\skills{Processing}{\fivenotes Image}
\skills{Java}{\threeskill}
\skills{Javacript}{\twoskill}
\skills{jQuery}{\twoskill}
%\skills{LAMP Stack}{\fournotes}
\skills{\LaTeX}{\twoskill}
%% \skills{LISP}{\onenote}
%\skills{Learning}{\fournotes Machine}
\skills{Maple}{\twoskill}
\skills{MATLAB}{\threeskill}
%\skills{MS Office}{\fivenotes}
%\skills{Networking}{\threenotes}
\skills{NumPy}{\threeskill}
\skills{OpenCV}{\threeskill}
\skills{PHP}{\oneskill}
\skills{Python}{\threeskill}
\skills{Django}{\twoskill}
\skills{SciPy}{\threeskill}
\skills{Scheme}{\threeskill}
\skills{SQL}{\twoskill}
\skills{Sys. Admin.}{\threeskill}
%% \skills{Visual Basic}{\fivenotes}
%\skills{Windows}{\fivenotes}
%% \skills{Wordpress}{\fournotes}

\end{itemize}

\end{multicols}
\vspace{-1em}
\begin{footnotesize}
  \oneskill Small-scale projects and/or assignments \hfill 
  \twoskill Multiple projects and/or experience teaching \hfill
  \threeskill Large-scale and/or multi-group projects
\end{footnotesize}

\section{Activities}
teaching, programming, open source software, system administration,
data visualization, Linux,
\href{https://soundcloud.com/malloc47}{music composition}

\null\vfill
\footnotesize{
  Online:  \href{http://cv.malloc47.com}{cv.malloc47.com} \hfill
  Résumé: \href{http://resume.malloc47.com}{resume.malloc47.com} \hfill 
  Source:  \href{https://github.com/malloc47/cv/tree/master}{github.com/malloc47/cv/}
}

%%\XeTeXpdffile ''cv.pdf'' page 1 scaled 800

\end{document}
