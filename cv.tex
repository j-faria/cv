\documentclass[a4paper,10pt]{article}

\usepackage{marvosym}
\usepackage{fontspec}
\usepackage{xunicode,xltxtra,url,parskip}
\RequirePackage{color,graphicx}
\usepackage[usenames,dvipsnames]{xcolor}
\usepackage[big]{layaureo}
\usepackage{fullpage}
\usepackage{supertabular}
\usepackage{titlesec}
\usepackage{multicol}
\usepackage{multirow}
\usepackage{longtable}
\usepackage{rotating}
\usepackage{ifthen}
\usepackage{hyperref}
\usepackage[absolute]{textpos}
\usepackage{enumitem}

\makeatletter
\renewcommand*{\@biblabel}[1]{\hfill[C#1]}
\makeatother

% Template obtained from http://www.cv-templates.info/2009/03/professional-cv-latex/

%Setup hyperref package, and colours for links
\definecolor{linkcolour}{rgb}{0,0.2,0.6}
\hypersetup{colorlinks,breaklinks,urlcolor=linkcolour, linkcolor=linkcolour}

%Color
\definecolor{lightg}{HTML}{999999}
\definecolor{medg}{HTML}{666666}
\definecolor{darkg}{HTML}{333333}

% Bullets
\definecolor{noteone}{HTML}{999999}
\definecolor{notetwo}{HTML}{848484}
\definecolor{notethree}{HTML}{424242}
\definecolor{notefour}{HTML}{212121}
\definecolor{notefive}{HTML}{000000}

\newcommand{\fivenotes}{%
	\textcolor{noteone}{\symbol{"2022}}
	\textcolor{notetwo}{\symbol{"2022}}
	\textcolor{notethree}{\symbol{"2022}}
	\textcolor{notefour}{\symbol{"2022}}
	\textcolor{notefive}{\symbol{"2022}}
}
\newcommand{\fournotes}{%
	\textcolor{noteone}{\symbol{"2022}}
	\textcolor{notetwo}{\symbol{"2022}}
	\textcolor{notethree}{\symbol{"2022}}
	\textcolor{notefour}{\symbol{"2022}}
	\textcolor{white}{\symbol{"2022}}
}
\newcommand{\threenotes}{%
	\textcolor{noteone}{\symbol{"2022}}
	\textcolor{notetwo}{\symbol{"2022}}
	\textcolor{notethree}{\symbol{"2022}}
	\textcolor{white}{\symbol{"2022}}
	\textcolor{white}{\symbol{"2022}}
}
\newcommand{\twonotes}{%
	\textcolor{noteone}{\symbol{"2022}}
	\textcolor{notetwo}{\symbol{"2022}}
	\textcolor{white}{\symbol{"2022}}
	\textcolor{white}{\symbol{"2022}}
	\textcolor{white}{\symbol{"2022}}
}
\newcommand{\onenote}{%
	\textcolor{noteone}{\symbol{"2022}}
	\textcolor{white}{\symbol{"2022}}
	\textcolor{white}{\symbol{"2022}}
	\textcolor{white}{\symbol{"2022}}
	\textcolor{white}{\symbol{"2022}}
}

\newcommand{\oneskill}{%
  \textcolor{white}{\symbol{"2022}}
  \textcolor{white}{\symbol{"2022}}
  \textcolor{notefive}{\symbol{"2022}}
}

\newcommand{\twoskill}{%
  \textcolor{white}{\symbol{"2022}}
  \textcolor{notethree}{\symbol{"2022}}
  \textcolor{notefive}{\symbol{"2022}}
}

\newcommand{\threeskill}{%
  \textcolor{noteone}{\symbol{"2022}}
  \textcolor{notethree}{\symbol{"2022}}
  \textcolor{notefive}{\symbol{"2022}}
}

%FONTS
% \defaultfontfeatures{Mapping=tex-text}
% \setmainfont[SmallCapsFont = Fontin SmallCaps]{Fontin}

\setromanfont [Ligatures={Common}, BoldFont={Linux Libertine Bold}, ItalicFont={Linux Libertine Italic}]{Linux Libertine}
\setsansfont [Ligatures={Common}, BoldFont={GeosansLight}, ItalicFont={GeosansLight}]{GeosansLight}
\setmonofont{GeosansLight} 

\font\lighttext=''Baskerville-Normal:color=787878'' at 10pt
\font\lighttextweb=''Baskerville-Normal:color=FF1493'' at 10pt

%CV Sections inspired by: 
%http://stefano.italians.nl/archives/26
\titleformat{\section}{\Large\scshape\raggedright\sffamily}{}{0em}{}[\titlerule]
%% \titlespacing{\section}{0pt}{-2pt}{0pt}

%-------------WATERMARK TEST [**not part of a CV**]---------------
\TPGrid[30mm,30mm]{30}{60}
%\setlength{\TPHorizModule}{30mm}
%\setlength{\TPVertModule}{\TPHorizModule}
%\textblockorigin{2mm}{0.65\paperheight}
\setlength{\parindent}{0pt}

\newcommand{\skill}{\textbf}
\newcommand{\institution}{\textsc}

% \def\bullet{\textcolor{medg}{\symbol{"00BB}}}
\def\div{\,\textbar{}\,}


\begin{document}
\pagestyle{empty}
% \font\fb=''[cmr10]''
\par{\centering {\Huge Jarrell \textsc{Waggoner} }\bigskip\par}

\section{Biographical Data}

\begin{tabular}{r p{12cm}}
  \textsc{Address:}	& Department of Computer Science and Engineering, University of South Carolina, Columbia, SC 29208 \\
  \textsc{Phone:}       & 847-261-4747\\
  \textsc{email:}       & \href{mailto:malloc47@gmail.com}{malloc47@gmail.com} \\
  % \textsc{Website:}	& \href{http://www.malloc47.com}{www.malloc47.com} \\
  \textsc{Citizenship:} & United States Citizen \\
\end{tabular}

\section{Web}
\begin{tabular}{r p{12cm}}
  \textsc{Twitter:}     & @malloc47 \\
  \textsc{Blog:}	& \href{http://www.malloc47.com}{www.malloc47.com} \\
  \textsc{github:}      & \href{http://www.github.com/malloc47}{http://www.github.com/malloc47}\\
  \textsc{Facebook:}    & \href{http://www.facebook.com/malloc47}{http://www.facebook.com/malloc47} \\
  \textsc{LinkedIn:}       & \href{http://www.linkedin.com/in/malloc47}{http://www.linkedin.com/in/malloc47} \\
\end{tabular}

\section{Education}
\begin{tabular}{r p{13.5cm}}	
  \textsc{Present} & Ph.D. Candidate in \textsc{Computer Science}, \textbf{University of South Carolina}\\
  \textsc{May} 2009 & Master of Engineering in \textsc{Computer Science}, \textbf{University of South Carolina}\\
  \textsc{May} 2006& Bachelor of Science in \textsc{Computer Science}, \textbf{Bryan College} \\
\textsc{May} 2004& Associate of Science in \textsc{Computer Science} \textbf{University of South Carolina at Lancaster} \\
\end{tabular}

\section{Experience}
\begin{longtable}{r|p{12cm}}
%\emph{Current} 
  \textsc{2011\textemdash{}Present}
  & \textbf{Research Assistant} funded by \textsc{AFOSR} \\
  &\emph{Materials Volume Segmentation}\\
  &\footnotesize{Developed segmentation methods for materials image
    volumes in \emph{Python+NumPy/SciPy} and \emph{MATLAB} at the
    \textsc{Computer Vision Lab} at \textsc{USC}. Managed the lab
    computer network and organized weekly lab meetings.  Created GUI
    interface using wxWidgets for assisted segmentation, and conducted
    large-scale evaluations on multiple datasets for metallic and
    biological materials.}
  \\\multicolumn{2}{c}{} \\
  \textsc{2011\textemdash{}Present}
  & \textbf{Project Manager} at \textsc{Palmetto Computer Labs}\\
  &\footnotesize{Assisted in planning the POSSCON conference. Managed
    the Open IT Lab and associated projects (Android
    Development). Provided software support for websites and managed
    projects.}
  \\\multicolumn{2}{c}{} \\
  \textsc{2011}
  & \textbf{Contractor} for \textsc{Elastic Vision Consulting} \\
  &\footnotesize{Created a parser and generator for XML medical
    records formats (CCR and CCD) in Java using JDOM, JAXB, SAX,
    Xerces, and Hibernate (HSQLDB), on an Axis2+Jetty6 driven server.}
  \\\multicolumn{2}{c}{} \\
  \textsc{2010\textemdash{}2011}
  & \textbf{Research Assistant} funded by \textsc{DARPA} \\
  &\emph{Video Event Recognition}\\
  &\footnotesize{Explored segmentation methods for video event
    recognition. Attended P.I. meetings in San Diego (2010) and
    Colorado (2011). Developed algorithms in \emph{Scheme} to process
    a corpus of thousands of videos extracted into over 3 million
    frames using a high-performance computing cluster.}
  \\\multicolumn{2}{c}{} \\
  \textsc{2009\textemdash{}2010}
  & \textbf{NSF Fellow} at the \textsc{USC Center for Digital Humanities} \\
  &\emph{Digital Collation}\\
  &\footnotesize{Created a \textsc{digital collation} application to
    handle automatic differencing of sub-textual inconsistencies among
    multiple copies of \emph{The Faerie Queene} by \textsc{Edmund
      Spenser} in \emph{MATLAB} to process tens of thousands of book
    page images.}
  \\\multicolumn{2}{c}{} \\
\end{longtable}

\section{Personal and Open Source Projects}
\begin{longtable}{r|p{13.5cm}}
  \textsc{term-do} & \footnotesize{A completion engine that is a hybrid of gnome-do and Emacs's ido-mode. Works on many tested VT100 terminal types and is built in C++. Includes full client/server architecture implemented with boost.interprocess and complete plugin system with bindings for multiple languages.}\\
  &\href{http://www.github.com/malloc47/term-do}{http://www.github.com/malloc47/term-do}\\
  \multicolumn{2}{c}{} \\
  \textsc{git-hq} & \footnotesize{A remote management system for git, coded in Python.}\\
  &\href{http://www.github.com/malloc47/git-hq}{http://www.github.com/malloc47/git-hq}\\
  \multicolumn{2}{c}{} \\
  \textsc{matscicut} & \footnotesize{An energy minimization framework for segmenting 3D materials volumes. Prototype of dissertation work, created in C++ using OpenCV libraries, with assorted MATLAB helper utilities.}\\
  &\href{http://www.github.com/malloc47/matscicut}{http://www.github.com/malloc47/matscicut}\\
  \multicolumn{2}{c}{} \\
\end{longtable}

%\section{Publications}
\nocite{waggoner:11}
\nocite{wang2011}
\nocite{temlyakov2010}
\nocite{zhang2010}
\renewcommand\refname{Publications}
\bibliography{cv}
\bibliographystyle{plainyr-rev}

% \section{Honors/Awards}
% \begin{tabular}{rll}
% 2011 & Graduate Student Day Presentation,  Second Place & \multirow{2}{*}{{\lighttext \textcolor{lightg}{USC}}}\\
% 2010 & Graduate Student Day Presentation,  Honorable Mention \\
% 2006 & Senior Computer Science Award & {\lighttext \textcolor{lightg}{Bryan College}}\\
% 2004 & Clara P. Hammond Award & \multirow{3}{*}{{\lighttext \textcolor{lightg}{\begin{turn}{-90}USCL\end{turn}}}} \\
% & Science and Mathematics Award \\
% & Highest Academic Average Award \\
% \end{tabular}

\section{Skills \& Languages}
\begin{multicols}{3}
\raggedcolumns
%\parbox{5cm}{
%\begin{enumerate}
\begin{itemize}
%\renewcommand{\labelenumi}{\textcolor{lightg}{[S\arabic{enumi}]}}
\renewcommand{\labelitemi}{\textcolor{lightg}{\symbol{"00BB}}}
\setlength{\itemsep}{1pt}
\setlength{\parskip}{0pt}
\setlength{\parsep}{0pt}
\item Bash \hfill \fournotes
\item Blender \hfill \threenotes
\item C/C++ \hfill \fournotes 
\item English \hfill \fivenotes
\item GIT/SVN/CVS \hfill \fournotes
\item GNU/Linux \hfill \fournotes
\item HTML/CSS \hfill \fournotes
% \item Image Processing \hfill \fournotes
\item Java \hfill \fivenotes
\item Javacript \hfill \threenotes
%\item LAMP Stack \hfill \fournotes
\item \LaTeX \hfill \fournotes
% \item Machine Learning \hfill \threenotes
% \item Maple \hfill \threenotes
\item MATLAB \hfill \fivenotes
%\item MS Office \hfill \fivenotes
%\item Networking \hfill \threenotes
\item OpenCV \hfill \fournotes
\item PHP \hfill \fournotes
\item Python \hfill \threenotes
\item Ruby \hfill \onenote
\item Scheme \hfill \fivenotes
% \item SQL \hfill \threenotes
% \item Sys. Admin. \hfill \threenotes
% \item Visual Basic \hfill \fivenotes
%\item Windows \hfill \fivenotes
\item Wordpress \hfill \fournotes
%\end{enumerate}
\end{itemize}
%}
\end{multicols}

\vspace{1em}

\begin{center}
  \parbox{10cm}{
    \onenote Small-scale projects and/or assignments \\
    \twonotes Implementation-specific experience \\
    \threenotes Quite familiar, used in larger projects \\
    \fournotes Extensive knowledge or experience teaching \\
    \fivenotes Used in context of large-scale and/or multi-group projects}
\end{center}

\vspace{1em}

\section{Interests and Activities}
Programming, Teaching, Mathematics\\
Open-source Software, Systems Administration, Linux\\
Typography, Music Composition

%%\XeTeXpdffile ''cv.pdf'' page 1 scaled 800

\end{document}
