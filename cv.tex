\documentclass[a4paper,10pt]{article}

\usepackage{marvosym}
\usepackage{fontspec}
\usepackage{xunicode,xltxtra,url,parskip}
\RequirePackage{color,graphicx}
\usepackage[usenames,dvipsnames]{xcolor}
\usepackage[big]{layaureo}
%\usepackage{fullpage}
\usepackage{supertabular}
\usepackage{titlesec}
\usepackage{multicol}
\usepackage{multirow}
\usepackage{longtable}
\usepackage{rotating}
\usepackage{ifthen}
\usepackage{hyperref}
\usepackage[absolute]{textpos}
\usepackage{enumitem}

\makeatletter
\renewcommand*{\@biblabel}[1]{\hfill[C#1]}
\makeatother

% Template obtained from http://www.cv-templates.info/2009/03/professional-cv-latex/

%Setup hyperref package, and colours for links
\definecolor{linkcolour}{rgb}{0,0.2,0.6}
\hypersetup{colorlinks,breaklinks,urlcolor=linkcolour, linkcolor=linkcolour}

%Color
\definecolor{lightg}{HTML}{999999}
\definecolor{medg}{HTML}{666666}
\definecolor{darkg}{HTML}{333333}

% Bullets
\definecolor{noteone}{HTML}{999999}
\definecolor{notetwo}{HTML}{848484}
\definecolor{notethree}{HTML}{424242}
\definecolor{notefour}{HTML}{212121}
\definecolor{notefive}{HTML}{000000}

\newcommand{\fivenotes}{%
	\textcolor{noteone}{\symbol{"2022}}
	\textcolor{notetwo}{\symbol{"2022}}
	\textcolor{notethree}{\symbol{"2022}}
	\textcolor{notefour}{\symbol{"2022}}
	\textcolor{notefive}{\symbol{"2022}}
}
\newcommand{\fournotes}{%
	\textcolor{noteone}{\symbol{"2022}}
	\textcolor{notetwo}{\symbol{"2022}}
	\textcolor{notethree}{\symbol{"2022}}
	\textcolor{notefour}{\symbol{"2022}}
	\textcolor{white}{\symbol{"2022}}
}
\newcommand{\threenotes}{%
	\textcolor{noteone}{\symbol{"2022}}
	\textcolor{notetwo}{\symbol{"2022}}
	\textcolor{notethree}{\symbol{"2022}}
	\textcolor{white}{\symbol{"2022}}
	\textcolor{white}{\symbol{"2022}}
}
\newcommand{\twonotes}{%
	\textcolor{noteone}{\symbol{"2022}}
	\textcolor{notetwo}{\symbol{"2022}}
	\textcolor{white}{\symbol{"2022}}
	\textcolor{white}{\symbol{"2022}}
	\textcolor{white}{\symbol{"2022}}
}
\newcommand{\onenote}{%
	\textcolor{noteone}{\symbol{"2022}}
	\textcolor{white}{\symbol{"2022}}
	\textcolor{white}{\symbol{"2022}}
	\textcolor{white}{\symbol{"2022}}
	\textcolor{white}{\symbol{"2022}}
}

\newcommand{\oneskill}{%
  \textcolor{white}{\symbol{"2022}}
  \textcolor{white}{\symbol{"2022}}
  \textcolor{notefive}{\symbol{"2022}}
}

\newcommand{\twoskill}{%
  \textcolor{white}{\symbol{"2022}}
  \textcolor{notethree}{\symbol{"2022}}
  \textcolor{notefive}{\symbol{"2022}}
}

\newcommand{\threeskill}{%
  \textcolor{noteone}{\symbol{"2022}}
  \textcolor{notethree}{\symbol{"2022}}
  \textcolor{notefive}{\symbol{"2022}}
}

%FONTS
% \defaultfontfeatures{Mapping=tex-text}
% \setmainfont[SmallCapsFont = Fontin SmallCaps]{Fontin}

\setromanfont [Ligatures={Common}, BoldFont={Linux Libertine Bold}, ItalicFont={Linux Libertine Italic}]{Linux Libertine}
\setsansfont [Ligatures={Common}, BoldFont={GeosansLight}, ItalicFont={GeosansLight}]{GeosansLight}
\setmonofont{GeosansLight} 

\font\lighttext=''Baskerville-Normal:color=787878'' at 10pt
\font\lighttextweb=''Baskerville-Normal:color=FF1493'' at 10pt

%CV Sections inspired by: 
%http://stefano.italians.nl/archives/26
\titleformat{\section}{\Large\scshape\raggedright\sffamily}{}{0em}{}[\titlerule]
%% \titlespacing{\section}{0pt}{-2pt}{0pt}

%-------------WATERMARK TEST [**not part of a CV**]---------------
\TPGrid[30mm,30mm]{30}{60}
%\setlength{\TPHorizModule}{30mm}
%\setlength{\TPVertModule}{\TPHorizModule}
%\textblockorigin{2mm}{0.65\paperheight}
\setlength{\parindent}{0pt}

\newcommand{\skill}{\textbf}
\newcommand{\institution}{\textsc}

% \def\bullet{\textcolor{medg}{\symbol{"00BB}}}
\def\div{\,\textbar{}\,}


\begin{document}
\pagestyle{empty}

% \font\fb=''[cmr10]''

\par{\centering {\Huge Jarrell \textsc{Waggoner} }\bigskip\par}

\section{Biographical Data}

\begin{tabular}{r p{3.5in}}
  \textsc{Address:}	& Department of Computer Science and Engineering, University of South Carolina, Columbia, SC 29208 \\
  \textsc{Phone:}     & 847-261-4747\\
  \textsc{email:}     & \href{mailto:malloc47@gmail.com}{malloc47@gmail.com} \\
  \textsc{Website:}	& \href{http://www.malloc47.com}{www.malloc47.com} \\
  \textsc{Citizenship:} & United States Citizen \\
\end{tabular}

\section{Research Interests}

Computer vision, segmentation, contour completion, perceptural grouping, document image analysis, event recognition, image processing.

\section{Education}
\begin{tabular}{r p{12cm}}
  \textsc{Present} & Ph.D. Candidate in \textsc{Computer Science}, \textbf{University of South Carolina}\\
  &% Dissertation: ``Combining Global Labeling and Local Relabeling for Metallic Image Segmentation'' |
  \small Advisor: Dr. Song \textsc{Wang} % | \normalsize \textsc{Gpa}: 3.85/4.0
%\hyperlink{grds}{\hfill | \footnotesize Detailed List of Exams}
  \\% &\\
  \textsc{May} 2009 & Master of Engineering in \textsc{Computer Science}, \textbf{University of South Carolina}\\
  &\normalsize \textsc{GPA}: 3.8/4.0 | \small\emph{magna cum laude}
%\hyperlink{grds}{\hfill | \footnotesize Detailed List of Exams}
\\% &\\
\textsc{May} 2006& Bachelor of Science in \textsc{Computer Science}, \textbf{Bryan College} \\
%& Thesis: ``Aspect-Oriented Programming in an Extreme Programming Environment''\\
&\normalsize 
%\textsc{GPA}: ?.?/4.0 | 
\small\emph{summa cum laude}
%\hyperlink{grds_cleli}{\hfill| \footnotesize Detailed List of Exams}
\\% &\\
\textsc{May} 2004& Associate of Science in \textsc{Computer Science}\\
&\textbf{University of South Carolina at Lancaster} \\
&\normalsize \textsc{GPA}: 4.0/4.0 | \small\emph{summa cum laude}\\
%\hyperlink{grds_cleli}{\hfill| \footnotesize Detailed List of Exams}
\end{tabular}

\section{Research Experience}
\begin{tabular}{r|p{11cm}}
%\emph{Current} 
\textsc{2011---Present}
& Research Assistant funded by \textsc{AFOSR} \\
&\emph{Materials Volume Segmentation}\\
&\footnotesize{Developed segmentation methods for materials image volumes.  Created GUI interface for assisted segmentation, and conducted large-scale evaluations on multiple datasets for metallic and biological materials.}
\\\multicolumn{2}{c}{} \\
\textsc{2010---2011}
& Research Assistant funded by \textsc{DARPA} \\
&\emph{Video Event Recognition}\\
&\footnotesize{Explored segmentation methods for video event recognition while working at the \textsc{Computer Vision Lab} at \textsc{USC}. Managed lab computer network and organize weekly lab meetings.  Attended P.I. meetings in San Diego (2010) and Colorado (2011). Visited \textsc{Purdue University} working with Dr.~Jeffrey Mark Siskind (Dec~2010---Jan~2011).}
\\\multicolumn{2}{c}{} \\
\textsc{2009---2010}
& NSF Fellow at the \textsc{Center for Digital Humanities} \\
&\emph{Digital Collation}\\
&\footnotesize{Created a \textsc{digital collation} application to handle automatic differencing of sub-textual inconsistencies among multiple copies of \emph{The Faerie Queene} by \textsc{Edmund Spenser}.}
\\\multicolumn{2}{c}{} \\
\end{tabular}

\pagebreak

\section{Teaching Experience}
\begin{longtable}{r|p{11cm}}
%\emph{Current} 
\textsc{2008--2009}
& GK-12 Fellow at \textsc{Crayton Middle School} \\
%in conjunction with the \textsc{Center for Teaching Excellence} \\
&\emph{Teaching 8\textsuperscript{th} Grade Science}\\
&\footnotesize{Served in Crayton Middle School, coordinating with the classroom instructor to enhance the science curriculum and activities in an 8\textsuperscript{th} grade science classroom. Subsequently coordinated and taught at the \textsc{GK-12 Institute for Teachers}, presenting the activities developed and delivered in the classroom.}
\\\multicolumn{2}{c}{} \\
\textsc{2007--2008, 2011} & Graduate Teaching Assistant at \textsc{USC} \\
&\emph{Teaching Software Development and Web Scripting}\\
&\footnotesize{Supervised CSCE~145 labs, covering software development with \textsc{Java}, and taught CSCE~102, covering \textsc{Javascript}, \textsc{HTML}, and \textsc{CSS}. Taught~CSCE~211 covering digital logic design.}\\\multicolumn{2}{c}{} \\
\textsc{Spring 2007} &  Instructor for \textsc{CSCE 204} at \textsc{USCL} \\
&\emph{Teaching Introductory Programming}\\
&\footnotesize{Hired as special faculty. Taught introductory Visual Basic for majors and non-majors. Selected textbooks, developed all course material, graded all assignments. Worked with Dr. Noni M. Bohonak}\\\multicolumn{2}{c}{} \\
\textsc{Fall 2006} & Camp Instructor for \textsc{USCL Arts and Sciences Adventure Camp} \\
&\emph{Teaching 5\textsuperscript{th}-8\textsuperscript{th} Grade Students}\\
&\footnotesize{Worked in collaboration with Dr. Dwayne Brown. One of two instructors teaching Math and Computer Science to grade school students.}\\\multicolumn{2}{c}{} \\
\textsc{2003--2007} & Professional Tutor at \textsc{USCL Academic Success Center} \\
&\emph{Tutoring High School and College Students}\\
&\footnotesize{Student and graduate tutor for college-level Mathematics, Computer Science, Physics, and English classes.}\\\multicolumn{2}{c}{} \\
\end{longtable}

%\section{Publications}
\nocite{waggoner:11}
\nocite{wang:11}
\nocite{temlyakov:10}
\nocite{zhang:10}
\nocite{waggoner:12}
\nocite{barbu:12}
\nocite{zhang:12}

\renewcommand\refname{Publications}
\bibliography{cv}
\bibliographystyle{plainyr-rev}

\section{Presentations}
\begin{enumerate}
\renewcommand{\labelenumi}{[P\arabic{enumi}] }
\item \emph{Combining Global Labeling and Local Relabeling for Metallic Image Segmentation}. Graduate Student Day Competition, Second Place. April 8, 2011.
\item \emph{Image Registration for Digital Collation}. Graduate Student Day Competition, Honorable Mention. April 2, 2010.
\item \emph{Aspect-Oriented Programming}. In CSCE 531. Guest lecture for Dr. Marco Valtorta. March 19, 2008.
\item \emph{Math 241}. Vector Calculus. Guest lecture for Dr. Dwayne Brown. April~23---26, 2007. 
\item \emph{Math 242}. Differential Equations. Guest lecture for Dr. Dwayne Brown. April~23---26, 2007. 
\end{enumerate}

%\section{Travel}
%\begin{enumerate}
%\renewcommand{\labelenumi}{[T\arabic{enumi}] }
%\item \emph{Mind's Eye PI Meeting}. DARPA Project P.I.~meeting. Denver, CO. January 20---21, 2011. 
%\item \emph{Tomography and its Applications to Materials Science and Non-Destructive Evaluation}. Organizd by M. De Graef, L. Drummy, J. Simmons, M. Comer, C. Bouman, and J. Knopp. Tech\^{}Edge, Dayton, Ohio. December 13---15, 2010.
%\item Visiting scholar. In collaboration with J. M. Siskind. Purdue University, West Lafayette, IN. December 6---22, 2010 \& January 5---16, 2011.
%\item \emph{Mind's Eye Kickoff Meeting}. DARPA Project P.I.~meeting. San Diego, CA. September 23---24, 2010. 
%\end{enumerate}


\section{Honors/Awards}
\begin{tabular}{rll}
2011 & Graduate Student Day Presentation,  Second Place & \multirow{2}{*}{{\lighttext \textcolor{lightg}{USC}}}\\
2010 & Graduate Student Day Presentation,  Honorable Mention \\
2006 & Senior Computer Science Award & {\lighttext \textcolor{lightg}{Bryan College}}\\
2004 & Clara P. Hammond Award & \multirow{3}{*}{{\lighttext \textcolor{lightg}{\begin{turn}{-90}USCL\end{turn}}}} \\
& Science and Mathematics Award \\
& Highest Academic Average Award \\
\end{tabular}

%\section{Professional Societies}

\section{Teaching}
\begin{center}
%\begin{tabular*}{0.75\textwidth}{r @{\hspace{0.5em}\textcolor{lightg}{\symbol{"00BB}}\hspace{0.5em}} l @{\extracolsep{\fill}} l }
\begin{tabular*}{0.75\textwidth}{r @{\hspace{0.5em}\textcolor{lightg}{\symbol{"00BB}}\hspace{0.5em}} l l c }
%\multicolumn{3}{r}{University of South Carolina}\\
Fall 2011 & CSCE 211 & Digital Logic Design & \multirow{4}{*}{{\lighttext \textcolor{lightg}{\begin{turn}{-90}USC\end{turn}}}} \\
Summer II 2008 & CSCE 102 & HTML/CSS/Javasript \\
Spring 2008 & CSCE 145 Lab & Java \\
Fall 2007 & CSCE 145 Lab & Java \\
\multicolumn{3}{r}{}\\
%\multicolumn{3}{r}{University of South Carolina at Lancaster}\\ \hline
Spring 2007 & CSCE 204 & Visual Basic & \multirow{3}{*}{{\lighttext \textcolor{lightg}{\begin{turn}{-90}USCL\end{turn}}}} \\
Spring 2007 & Math 241 \& Math 242 & Maple \\
\multicolumn{1}{c}{}& (Guest Lecture) & \\
\end{tabular*}
\end{center}

\section{Skills \& Languages}
\begin{multicols}{3}
\raggedcolumns
%\parbox{5cm}{
%\begin{enumerate}
\begin{itemize}
%\renewcommand{\labelenumi}{\textcolor{lightg}{[S\arabic{enumi}]}}
\renewcommand{\labelitemi}{\textcolor{lightg}{\symbol{"00BB}}}
\setlength{\itemsep}{1pt}
\setlength{\parskip}{0pt}
\setlength{\parsep}{0pt}
\item Assembly \hfill \twonotes
\item Bash \hfill \fournotes
\item Blender \hfill \threenotes
\item C/C++ \hfill \fournotes 
\item English \hfill \fivenotes
\item GIT/SVN/CVS \hfill \fournotes
\item GNU/Linux \hfill \fournotes
\item HTML/CSS \hfill \fournotes
%\item Image Processing \hfinll \fivenotes
\item Java \hfill \fivenotes
\item Javacript \hfill \threenotes
%\item LAMP Stack \hfill \fournotes
\item \LaTeX \hfill \threenotes
\item LISP \hfill \onenote
%\item Machine Learning \hfill \fournotes
\item Maple \hfill \threenotes
\item MATLAB \hfill \fivenotes
%\item MS Office \hfill \fivenotes
%\item Networking \hfill \threenotes
\item OpenCV \hfill \fournotes
\item PHP \hfill \fournotes
\item Python \hfill \threenotes
\item Scheme \hfill \fivenotes
\item SQL \hfill \threenotes
\item Sys. Admin. \hfill \threenotes
\item Visual Basic \hfill \fivenotes
%\item Windows \hfill \fivenotes
\item Wordpress \hfill \fournotes
%\end{enumerate}
\end{itemize}
%}
\end{multicols}

\vspace{1em}

\begin{center}
\parbox{12cm}{
\onenote Some familiarity, small-scale projects and assignments \\
\twonotes Implementation-specific experience \\
\threenotes Quite familiar, used in limited settings as part of larger projects \\
\fournotes Extensive knowledge or experience teaching \\
\fivenotes Used in context of large scale, multi-group projects}
\end{center}

% \vspace{1em}
% 
% \section{Interests and Activities}
% Programming, Teaching, Mathematics\\
% Open-source Software, System Administration, Linux\\
% Typography, Music Composition

%%\XeTeXpdffile ''cv.pdf'' page 1 scaled 800

\end{document}
